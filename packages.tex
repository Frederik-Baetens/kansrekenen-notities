% Voor subfiles
\usepackage{subfiles}

% Voor todo's
\usepackage{todonotes}

% Voor wiskunde
\usepackage{amsmath}
\usepackage{amsfonts}
\usepackage{amssymb}
\usepackage{amsthm}

% Voor urls
\usepackage{hyperref}

% Een mooier bestand met wiskunde ondersteuning
\usepackage{libertine}
\usepackage[libertine]{newtxmath}

% Om het totaal aantal pagina's te tellen
\usepackage{lastpage}
\usepackage{afterpage}

% Om figuren op de juiste plaats te krijgen
\usepackage{float}

% Voor frames
\usepackage{mdframed}

% Om de marges aan te passen
\usepackage[left=2cm,right=2cm,top=2cm,bottom=2cm]{geometry}

% Voor mooiere breuken
\usepackage{nicefrac}

% Voor intervallen
\usepackage{interval}%   

%indices
\usepackage{makeidx}
\makeindex

