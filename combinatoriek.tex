\documentclass[main.tex]{subfiles}
\begin{document}

\chapter{Combinatoriek}
\label{cha:combinatoriek}


\section{Variaties}
\label{sec:variaties}

\begin{de}
  Een \term{variatie} van $p\in \mathbb{N}$ \textbf{verschillende} elementen uit $n\in \mathbb{N}$ elementen (met $p \le n$) is een \textbf{geordend} $p$-tal van \textbf{verschillende elementen} gekozen uit een gegeven verzameling van $n$ elementen.    
\end{de}

\begin{st}
  Het aantal variaties van $p$ elementen uit $n$ is gelijk aan $V_{v}^{p}$.
  \[
  V_{p}^{n} = \prod_{i=0}^{p-1}(n-i)
  \]

  \begin{proof}
    We moeten een rij van lengte $p$ vormen met $n$ elementen.
    Voor de eerste plaats is er vrije keuze, dus $n$ elementen.
    Voor de tweede plaats is er keuze uit de overgebleven $n-1$ elementen.
    Voor de $i$-de plaats is er dan keuze uit de overgebleven $n-1$ elementen.
    er zijn dus $\prod_{i=0}^{p-1}(n-p)$ mogelijke manieren om $p$ elementen te kiezen uit $n$ zonder herhalingen.
  \end{proof}
\end{st}

\begin{ei}
  \[ V_{n}^{0} = 1 \quad\text{en}\quad V_{n}^{n} = n! \]

  \begin{proof}
    \[ V_{n}^{0} = \prod_{i=0}^{0-1}(n-i) = 1 \quad\text{ en }\quad V_{n}^{n} = \prod_{i=0}^{n-1}(n-i) = n!\]
  \end{proof}
\end{ei}

\begin{de}
  Een \term{herhalingsvariatie} van $p\in \mathbb{N}$ elementen uit $n\in \mathbb{N}$ elementen is een \textbf{geordend} $p$-tal van \textbf{niet noodzakelijk verschillende} elementen gekozen uit een gegeven verzameling van $n$ elementen.  
\end{de}

\begin{st}
  Het aantal herhalingsvariaties van $p$ elementen uit $n$ is gelijk aan $\overline{V}_{v}^{p}$.
  \[
  \overline{V}_{p}^{n} = n^{p}
  \]

  \begin{proof}
    Voor elk element van het $p$-tal is er keuze uit $n$ elementen. 
    Er zijn dus $n^{p}$ mogelijke herhalingsvariaties van $p$ elementen uit $n$.
  \end{proof}
\end{st}

\section{Permutaties}
\label{sec:permutaties}

\begin{de}
  Een \term{permutatie} van $n\in \mathbb{N}$ elementen is een variatie van $n$ uit $n$ elementen. 
\end{de}

\begin{opm}
  Soms wordt een permutatie ook beschreven als een bijectie van een eindige verzameling naar zichzelf.
  Deze noties komen overeen in de zin dat de variatie beschreven in bovenstaande definitie een beschrijving geeft van de bijectie in de andere definitie.
\end{opm}

\begin{de}
  Een \term{herhalingspermutatie} van $n\in \mathbb{N}$ elementen waarvan $p_{i}$ elementen telkens tot soort $i$ behoren (met $\sum_{i=1}^{r} p_{i} = n$) is een \textbf{geordend} $n$-tal van elementen waarvan de $i$-de $p_{i}$ elementen telkens tot soort $i$ behoren.
\end{de}

\begin{st}
  Het aantal herhalingspermutaties van $p$ elementen uit $n$ is gelijk aan $\overline{P}_{n}^{p_{1},p_{2},\dotsc,p_{r}}$.
  \[ \overline{P}_{n}^{p_{1},p_{2},\dotsc,p_{r}} = \frac{n!}{\prod_{i=1}^{r}p_{i}} = \binom{n}{p_{1}\ p_{2}\ \dotsb\ p_{r}} \]
\extra{bewijs}
\end{st}

\section{Combinaties}
\label{sec:combinaties}

\begin{de}
  Een \term{combinatie} van $p$ elementen uit $n$ is een deelverzameling van die $n$ elementen.
\end{de}

\begin{opm}
  Bij een combinatie speelt de volgorde van de elementen dus geen rol.
\end{opm}

\begin{st}
  Het aantal combinaties van $p$ elementen uit $n$ is $C_{n}^{p}$.
  \[ C_{n}^{p} = \frac{n!}{p!(n-p)!} = \binom{n}{p} \]

  \begin{proof}
    Het aantal variaties $V_{n}^{p}$ van $p$ elemenen uit $n$, (waar de volgorde wel een rol speelt) is teveel.
    Het is zelfs precies $P_{p}$ keer teveel want we kunnen de elementen in het $n$ tal nog permuteren.
    \[ C_{n}^{p} = \frac{V_{n}^{p}}{P_{p}} = \frac{n!}{p!(n-p)!} \]
  \end{proof}
\end{st}

\begin{opm}
  $\binom{n}{p}$ lezen we als ``$n$ kies $p$''.
\end{opm}

\begin{ei}
  \[ C_{n}^{n} = C_{n}^{0} = 1 \]

  \begin{proof}
    \[ C_{n}^{n} = \frac{n!}{n!(n-n)!} = 1 = \frac{n!}{0!(n-0!)} = C_{n}^{0} \]
  \end{proof}
\end{ei}

\begin{ei}
  \[ C_{n}^{p} = C_{n}^{n-p} \]

  \begin{proof}
    \[
    \begin{array}{rll}
      C_{n}^{n-p} &= \frac{n!}{(n-p)!(n-(n-p))!} \\
                &= \frac{n!}{(n-p)!(n-n+p)!} \\
                &= \frac{n!}{p!(n-p)!} &= C_{n}^{p}
    \end{array}
    \]
  \end{proof}
\end{ei}

\begin{ei}
  De \term{formule van Pascal}.
  \[ C_{n}^{p} + C_{n}^{p+1} = C_{n+1}^{p+1} \]

  \begin{proof}
    \[
    \begin{array}{rll}
      C_{n}^{p} + C_{n}^{p+1} &= \frac{n!}{p!(n-p)!} + \frac{n!}{(p+1)!(n-(p+1))!} \\
                           &= \frac{n!}{p!(n-p)(n-p-1)!} + \frac{n!}{(p+1)p!(n-p-1))!} \\
                           &= \frac{n!(p+1)+ n!(n-p)}{(p+1)p!(n-p)(n-p-1)!} \\
                           &= \frac{(n+1)n!}{(p+1)!(n-p)!} \\
                           &= \frac{(n+1)!}{(p+1)!((n+1)-(p+1)!} &= C_{n+1}^{p+1}\\
    \end{array}
    \]
  \end{proof}
\end{ei}

\begin{de}
  Een \term{herhalingscombinatie} van $p$ elementen uit $n$ is een \textbf{ongeordend} $p$-tal van elementen waarbij herhaling mogelijk is.
\end{de}

\begin{opm}
  De notie van een ``ongeordend $p$-tal'' is ietwat vaag.
  In feite hebben we nood aan een structuur die geen orde oplegt en herhaling van elementen toelaat.
  Noch een geordend $p$-tal, noch een verzameling is hiervoor dus bruikbaar.
\end{opm}

\begin{st}
  Het aantal herhalingscombinaties van $p$ elementen uit $n$ is $\overline{C}_{n}^{p}$.
  \[ \overline{C}_{n}^{p} = C_{n+p-1}^{p} = \binom{n+p-1}{p} \]
\end{st}

\subsection{Binomium van Newton}
\label{sec:binomium-van-newton}

\begin{st}
  Het \term{binomium van Newton}.
  \[ (a+b)^{n} = \sum_{i=0}^{n} = \binom{n}{i} a^{n-i}b^{i} \]
\extra{bewijs}
\end{st}

\begin{de}
  De co\"efficienten $\binom{n}{i}$ worden daarom ook de \term{binomiaalco\"efficienten} genoemd.
\end{de}

\subsection{Multinomiale ontwikkeling}
\label{sec:mult-ontw}

\begin{st}
  De \term{multinomiale ontwikkeling}.
  \[ \left(\sum_{i=1}^{k}\right)^{n} = \sum_{\sum_{i=1}^{k}n_{i}=n}\left(\binom{n}{n_{1}\ n_{2}\ \dotsb\ n_{k}}\prod_{i=1}^{k}x_{i}^{n_{i}}\right) \]
\end{st}

\begin{de}
  De co\"efficienten $\binom{n}{n_{1}\ n_{2}\ \dotsb\ n_{k}}$ worden daarom ook de \term{multinomiaalco\"efficienten} genoemd.
\end{de}

\subsection{Het aantal deelverzamelingen}
\label{sec:het-aant-deelv}

\begin{st}
  Het aantal deelverzamelingen $|\mathcal{P}(V)|$ van een verzameling $V$ is $2^{|V|}$.

  \begin{proof}
    Het aantal deelverzamelingen van $i$ elementen van $V$ is $C_{|V|}^{i}$.
    Het totaal aantal deelverzamelingen is dan $\sum_{i=0}^{|V|}C_{|V|}^{i}$.
    Gebruik nu het binomium van newton met $a=b=1$ om het gezochte resultaat te bekomen.
    \[ \sum_{i=0}^{|V|}C_{|V|}^{i} = \sum_{i=0}^{|V|}\binom{|V|}{i} = \sum_{i=0}^{|V|}\binom{|V|}{i}1^{(|V|-i)}1^{i} = (1+1)^{|V|} = 2^{|V|} \]
  \end{proof}
\end{st}


\end{document}
