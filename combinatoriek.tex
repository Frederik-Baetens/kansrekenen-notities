\documentclass[main.tex]{subfiles}
\begin{document}

\chapter{Combinatoriek}
\label{cha:combinatoriek}

\section{Variaties}
\label{sectie:variaties}

\begin{de}
  Een \term{variatie} van $p\in \mathbb{N}$ elementen uit $n\in \mathbb{N}$ elementen (met $p \le n$) is een \textbf{geordend} $p$-tal van \textbf{verschillende elementen} gekozen uit een gegeven verzameling van $n$ elementen.    
\end{de}

\begin{de}
  Twee variaties zijn gelijk als zowel de orde van als de elementen in het $p$-tal gelijk zijn.
\end{de}

\begin{st}
  Het aantal variaties van $p$ elementen uit $n$ is gelijk aan $V_{v}^{p}$.
  \[
  V_{p}^{n} = \prod_{i=0}^{p-1}(n-p)
  \]

  \begin{proof}
    We moeten een rij van lengte $p$ vormen met $n$ elementen.
    Voor de eerste plaats is er vrije keuze, dus $n$ elementen.
    Voor de tweede plaats is er keuze uit de overgebleven $n-1$ elementen.
    Voor de $i$-de plaats is er dan keuze uit de overgebleven $n-1$ elementen.
    er zijn dus $\prod_{i=0}^{p-1}(n-p)$ mogelijke manieren om $p$ elementen te kiezen uit $n$ zonder herhalingen.
  \end{proof}
\end{st}

\end{document}
