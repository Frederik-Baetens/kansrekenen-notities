\documentclass[main.tex]{subfiles}
\begin{document}

\chapter{Voorkennis}
\label{cha:voorkennis}


\section{Verzamelingen}
\label{sec:verzamelingen}

\subsection{Basisbegrippen}
\label{sec:basisbegrippen}

\begin{de}
  Een \term{verzameling} is een geheel van onderling verschillende, ongeordende objecten. Deze objecten noemt men de elementen van de verzameling.
de 
\end{de}

\begin{de}
  Een formele beschrijving van een verzameling met behulp van een predikaat $p$ ziet er als volgt uit.
  \[ \{x\ |\ p(x)\} \]
  Dit is de verzameling van all elementen die aan het predikaat $p$ voldoen.
\end{de}

\begin{de}
  Twee verzamelingen $A$ en $B$ zijn \term{gelijk} als en slechts als ze dezelfde elementen bevatten. 
  \[ A = B \Leftrightarrow \forall x:\ x \in A \Leftrightarrow x \in B \]
\end{de}

\begin{st}
  De \term{transitiviteit} van '$=$': Gegeven drie willekeurige verzamelingen $A$, $B$ en $C$.
  \[ (A = B) \wedge (B = C) \Rightarrow A = C \]
  \begin{proof}
    \[
    \begin{array}{cll}
      A = B \wedge B = C &\Leftrightarrow (\forall x:\ x \in A \Leftrightarrow x \in B) \wedge (\forall x:\ x \in B \Leftrightarrow x \in C) &\\
      &\Rightarrow \forall x:\ ((x \in A \Leftrightarrow x \in B) \wedge (x \in B \Leftrightarrow x \in C)) &\\
      &\Leftrightarrow \forall x:\ (x \in A \Leftrightarrow x \in C) &\Leftrightarrow  A = C\\
    \end{array}
    \]
  \end{proof}
\end{st}

\begin{de}
  Een verzameling $A$ is \term{een deelverzameling} van een verzameling $B$ als en slechts als $B$ alle elementen van $A$ bevat.
  \[ A \subseteq B \Leftrightarrow \forall x:\ x \in A \Rightarrow x \in B\]
\end{de}

\begin{st}
  De \term{anti-symmetrie} van '$\subseteq$': Gegeven twee willekeurige verzamelingen $A$ en $B$.
  \[ A \subseteq B \wedge B \subseteq A  \Leftrightarrow A = B \]
  \begin{proof}
    \[
    \begin{array}{cll}
      A \subseteq B \wedge B \subseteq A &\Leftrightarrow (\forall x:\ x \in A \Rightarrow x \in B) \wedge (\forall x:\ x \in B \Rightarrow x \in A) &\\
      & \Leftrightarrow \forall x:\ ((x \in A \Rightarrow x \in B) \wedge (x \in B \Rightarrow x \in A)) &\\
      & \Leftrightarrow \forall x:\ x \in A \Leftrightarrow x \in B &\Leftrightarrow A = B  \\
    \end{array}
    \]
  \end{proof}
\end{st}

\begin{st}
  De \term{transitiviteit} van '$\subseteq$': Gegeven drie willekeurige verzamelingen $A$, $B$ en $C$.
  \[ A \subseteq B \wedge B \subseteq C  \Leftrightarrow A \subseteq C \]
  \begin{proof}
    \[
    \begin{array}{cll}
      A \subseteq B \wedge B \subseteq C &\Leftrightarrow (\forall x:\ x \in A \Rightarrow x \in B) \wedge (\forall x:\ x \in B \Rightarrow x \in C) &\\
      & \Rightarrow \forall x:\ ((x \in A \Rightarrow x \in B) \wedge (x \in B \Rightarrow x \in C)) &\\
      & \Leftrightarrow \forall x:\ x \in A \Rightarrow x \in C &\Leftrightarrow A \subseteq C  \\
    \end{array}
    \]
  \end{proof}
\end{st}

\begin{de}
  Een verzameling $A$ is een \term{strikte deelverzameling} van een verzameling $B$ als en slechts als $A$ een deelverzameling is van $B$ en niet gelijk is aan $B$.
  \[ A \subsetneq B \Leftrightarrow A \subseteq B \wedge a \neq B \]
\end{de}

\begin{de}
  De \term{universele verzameling} $U$ is de verzameling van alle mogelijke elementen waarvan sprake is.
  \[ U = \{ x\ |\ true\} \]
\end{de}

\begin{st}
  Elke verzameling $A$ is een deelverzameling van het universum $U$.
  \[ A \subseteq U \]
  \begin{proof}
    Inderdaad. Kies een willekeurige verzameling $A$. 
    Elk element van $A$ zit ook in $U$.
    \[
    \forall x:\ x \in A \Rightarrow x \in U
    \]
  \end{proof}
\end{st}

\begin{de}
  De \term{lege verzameling} $\emptyset$ is de verzameling die geen enkel element bevat.
\end{de}

\begin{st}
  De lege verzameling $\emptyset$ is een deelverzameling van elke verzameling.
  \begin{proof}
    Inderdaad. Kies een willekeurige verzameling $A$.
    Elk element van $\emptyset$ (geen enkel element) zit ook in $A$.
    \[
    \forall x:\ x \in \emptyset \Rightarrow x \in A
    \]
  \end{proof}
\end{st}

\begin{de}
  Een \term{singleton} is een verzameling met precies \'e\'en element.
\end{de}

\subsection{De algebra van verzamelingen}
\label{sec:de-algebra-van-verzamelingen}
\subsubsection{Unie}
\label{sec:unie}

\begin{de}
  De \term{unie} $A \cup B$ \term{van twee verzamelingen} $A$ en $B$ is de verzameling die zowel de elementen van $A$ als de elementen van $B$ bevat.
  \[ A \cup B = \{ x\ |\ x \in A \vee x \in B\} \]
\end{de}

\begin{ei}
  De \term{unie is commutatief}.
  \[ A \cup B = B \cup A \]
  \begin{proof}
    $A \cup B = \{ x\ |\ x \in A \vee x \in B\} = \{ x\ |\ x \in B \vee x \in A\} = B \cup A$
  \end{proof}
\end{ei}

\begin{ei}
  De \term{unie is idempotent}
  \[ A \cup A = A \]
  \begin{proof}
    $A \cup A = \{ x\ |\ x \in A \vee x \in A \} = \{ x\ |\ x \in A \} = A$
  \end{proof}
\end{ei}

\begin{st}
  Elke verzameling $A$ is een deelverzameling van elke unie $A \cup B$ van die verzameling met een andere verzameling $B$.
  \[ A \subseteq A \cup B \]
  
  \begin{proof}
    $\forall x:\ x \in A \Leftrightarrow x \in A \vee x \in B$
  \end{proof}
\end{st}

\begin{st}
  \[ A \subseteq B \Leftrightarrow A \cup B = B \]

  \begin{proof}
    $\{ x\ |\ x \in A \vee x \in B\} = B \Leftrightarrow \forall a\in A:\ a\in B$
  \end{proof}
\end{st}

\begin{st}
  De unie is \term{associatief}
  \[ A \cup (B \cup C) = (A \cup B) \cup C \]

  \begin{proof}
    $A \cup \{ x\ |\ x \in B \vee x \in C \} = \{ x\ |\ x \in A \vee x \in B \vee x \in C\} = \{ x\ |\ x \in A \vee x \in B\} \cup C$
  \end{proof}
\end{st}

\begin{st}
  De \term{identiteitswet} voor de unie
  \[ A \cup \emptyset = A \]
  
  \begin{proof}
    $A \cup \emptyset = \{ x\ |\ x \in A \vee x \in \emptyset \} = A$
  \end{proof}
\end{st}

\begin{st}
  De \term{nulwet} voor de unie
  \[ A \cup U = U \]
  \begin{proof}
    $A \cup U = \{ x\ |\ x \in A \vee x \in U\} = \{ x\ |\ x \in A \vee true \} = U$
  \end{proof}
\end{st}

\subsubsection{Doorsnede}
\label{sec:doorsnede}

\begin{de}
  De \term{doorsnede} $A \cap B$ \term{van twee verzamelingen} $A$ en $B$ is de verzamling die enkel de elementen bevat die zowel in $A$ als in $B$ zitten.
  \[ A \cap B = \{ x\ |\ x \in A \wedge x \in B\} \]
\end{de}

\begin{ei}
  De \term{doorsnede} is \term{commutatief}.
  \[ A \cap B = B \cap A \]
  \begin{proof}
    $A \cap B = \{ x\ |\ x \in A \wedge x \in B\} = \{ x\ |\ x \in B \wedge x \in A\} = B \cap A$
  \end{proof}
\end{ei}

\begin{ei}
  De \term{doorsnede} is idempotent
  \[ A \cap A = A \]
  \begin{proof}
    $A \cap A = \{ x\ |\ x \in A \wedge x \in A\} = \{ x\ |\ x \in A \} = A$
  \end{proof}
\end{ei} 

\begin{st}
  De doorsnede $A \cap B$ is een deelverzameling van $A$.
  \[ A \cap B \subseteq A \]
  \begin{proof}
    $A \cap B = \{ x\ |\ x \in A \wedge x \in B \} \subseteq \{ x\ |\ x \in A \} = A  $
  \end{proof}
\end{st}

\begin{st}
  \[ A \subseteq B \Leftrightarrow A \cap B = A \]
  \begin{proof}
    $\forall x:\ (x\in A \Rightarrow x \in B) \Leftrightarrow \{ x\ |\ x \in A \wedge x \in B \} = A$
  \end{proof}
\end{st}

\begin{st}
  De \term{identiteitswet} voor de doorsnede
  \[ A \cap U = A \]
  \begin{proof}
    $A \cap U = \{ x\ |\ x \in A \wedge x \in U\} = \{ x\ |\ x \in A \wedge true \} = A$
  \end{proof}
\end{st}

\begin{st}
  De \term{nulwet} voor de doorsnede
  \[ A \cap \emptyset = \emptyset \]
  \begin{proof}
    $A \cap \emptyset = \{ x\ |\ x \in A \wedge x \in \emptyset \} = \{ x\ |\ x \in A \wedge false \} = \emptyset$
  \end{proof}
\end{st}
\begin{de}
  Twee verzamelingen $A$ en $B$ zijn \term{disjunct} als en slechts als ze geen gemeenschappelijke elementen hebben.
  \[ A \cap B = \emptyset \]
\end{de}

\begin{st}
  De eerste \term{absorptiewet}.
  \[ A \cup ( A \cap B ) = A\]

  \begin{proof}
    $A \cup ( A \cap B ) = A \cup \{ x\ |\ x \in A \wedge x \in B\} = \{ x\ |\ x\in A \vee (x \in A \wedge x \in B) \} = \{ x\ |\ x \in A \} = A$
  \end{proof}
\end{st}

\begin{st}
  De tweede \term{absorptiewet}.
  \[ A \cap ( A \cup B ) = A\]

  \begin{proof}
    $A \cap ( A \cup B ) = A \cap \{ x\ |\ x \in A \vee x \in B\} = \{ x\ |\ x\in A \wedge (x \in A \vee x \in B) \} = \{ x\ |\ x \in A \} = A$
  \end{proof}
\end{st}

\begin{st}
  De \term{doorsnede} is distributief ten opzichte van de unie.
  \[ A \cap ( B \cup C ) = (A \cap B) \cup (A \cap C) \]

  \begin{proof}
    \[
    \begin{array}{rll}
      A \cap ( B \cup C ) &= A \cap \{ x\ |\ x \in B \vee x \in C \}&\\
                          &= \{ x\ |\ x \in A \wedge (x \in B \vee x \in C) \}&\\
                          &= \{ x\ |\ (x \in A \wedge x\in B) \vee (x \in A \wedge x \in C) \} &= (A \cap B) \cup (A \cap C)      
    \end{array}
    \]
  \end{proof}
\end{st}

\begin{st}
  De \term{unie} is distributief ten opzichte van de doorsnede.
  \[ A \cup ( B \cap C ) = (A \cup B) \cap (A \cup C) \]

  \begin{proof}
    \[
    \begin{array}{rll}
      A \cup ( B \cap C ) &= A \cup \{ x\ |\ x \in B \wedge x \in C \}&\\
                          &= \{ x\ |\ x \in A \vee (x \in B \wedge x \in C) \}&\\
                          &= \{ x\ |\ (x \in A \vee x\in B) \wedge (x \in A \vee x \in C) \} &= (A \cup B) \cap (A \cup C)      
    \end{array}
    \]
  \end{proof}
\end{st}

\subsubsection{Complement}
\label{sec:complement}

\begin{de}
  Het \term{complement} van een verzameling $A$ ten opzichte van de universele verzameling $U$ is de verzameling van alle elementen die niet in $A$ zitten, maar wel in $U$.
  \[ A^{c} = \{ x\ |\ x \not\in A \} \]
  Andere notaties voor het complement zijn $A'$, $\overline{A}$. 
\end{de}

\begin{st}
  Het complement van het complement van een verzameling is opnieuw de originele verzameling.
  \[ (A^{c})^{c} = A\]
  \begin{proof}
    $A^{c^{c}} = \{ x\ |\ x \not\in A^{c} \} = \{ x\ |\ x \in A \} = A$
  \end{proof}
\end{st}

\begin{st}
  De \term{complementaire wet} voor de unie.\\
  De unie van een verzameling en haar complement is het universum.
  \[ A \cup A^{c} = U \]
  \begin{proof}
    $A \cup A^{c} = \{ x\ |\ x \in A \vee x \in A^{c}\} = \{ x\ |\ true \} = U$
  \end{proof}
\end{st}

\begin{st}
  De \term{complementaire wet} voor de doorsnede.\\
  De doorsnede van een verzameling en haar complement is leeg..
  \[ A \cap A^{c} = \emptyset \]

  \begin{proof}
    $A \cap A^{c} = \{ x\ |\ x \in A \wedge x \in A^{c} \} = \{ x\ |\ false \}= \emptyset$
  \end{proof}
\end{st}

\begin{st}
  De eerste \term{wet van De Morgan}.
  \[ (A \cup B)^{c} = A^{c} \cap B^{c} \]

  \begin{proof}
    \[
    \begin{array}{rll}
    (A \cup B)^{c} &= \{ x\ |\ x \not\in (A\cup B) \}&\\
                   &= \{ x\ |\ \neg(x \in A \vee x \in B) \}&\\
                   &= \{ x\ |\ (x \not\in A)\wedge (x \not\in B) \}&\\
                   &= A^{c} \cap B^{c}
    \end{array}
    \]
  \end{proof}
\end{st}

\begin{st}
  De tweede \term{wet van De Morgan}.
  \[ (A \cap B)^{c} = A^{c} \cup B^{c} \]

  \begin{proof}
    \[
    \begin{array}{rll}
      (A \cap B)^{c} &= \{ x\ |\ x \not\in (A\cap B) \}&\\
                     &= \{ x\ |\ \neg(x \in A \wedge x \in B) \}&\\
                     &= \{ x\ |\ (x \not\in A)\vee (x \not\in B) \}&\\
                     &= A^{c} \cup B^{c}
    \end{array}
    \]
  \end{proof}
\end{st}

\subsubsection{Verschil}
\label{sec:verschil}

\begin{de}
  Het \term{verschil} van een verzameling $A$ met een andere verzameling $B$ is de verzameling van alle elementen van $A$ die niet in $B$ zitten.
  \[ A \setminus B = \left\{ x\ |\ x \in A \wedge x \not\in B \right\} \]
\end{de}

\begin{pr}
  Voor twee verzamelingen $A$ en $B$ geldt dat zowel de doorsnede als de verschillen onderling disjunct zijn.
  \[ 
  \begin{array}{rl}
    (1) & (A \cap B) \cap (A \setminus B) = \emptyset\\
    (2) & (A \setminus B) \cap (B \setminus A) = \emptyset\\
    (3) & (B \setminus A) \cap (A \cap B) = \emptyset\\
  \end{array}
  \]
  \begin{proof}
    Bewijs elk deel afzonderlijk:
    \begin{itemize}
    \item 
      \[
      \begin{array}{rll}
        (A \cap B) \cap (A \setminus B) &= \{ x\ |\ x \in A \wedge x \in B\} \cap \{ x\ |\ x \in A \wedge x \not\in B \}&\\
        &= \{ x\ |\ (x \in A \wedge x \in B)\wedge (x \in A \wedge x \not\in B) \}&\\
        &= \{ x\ |\ (x \in B)\wedge (x \not\in B) \}&\\
        &= \{ x\ |\ false \} = \emptyset
      \end{array}
      \]
    \item
      \[
      \begin{array}{rll}
        (A \setminus B) \cap (B \setminus A) &= \{ x\ |\ x \in A \wedge x \not\in B \} \cap \{ x\ |\ x \in B \wedge x \not\in A \}&\\
                                             &= \{ x\ |\ (x \in A \wedge x \not\in B) \wedge (x \in B \wedge x \not\in A) \}&\\
                                             &= \{ x\ |\ false \}&\\
                                             &= \emptyset
      \end{array}
      \]
    \item 
      \[
      \begin{array}{rll}
        (B \setminus A) \cap (A \cap B) &= \{ x\ |\ x \in B\wedge x \not\in A \} \cap \{ x\ |\ x \in A \wedge x \in B\}&\\
        &= \{ x\ |\ (x \in B \wedge x \not\in A) \wedge (x \in A \wedge x \in B) \}&\\
        &= \{ x\ |\ (x \in A)\wedge (x \not\in A) \}&\\
        &= \{ x\ |\ false \} = \emptyset
      \end{array}
      \]
    \end{itemize}
  \end{proof}
\end{pr}


\begin{st}
  Het verschil van twee verzamelingen kan worden herschreven als de doorsnede met het complement.
  \[ A \setminus B = A \cap B^{c} \]
  \begin{proof}
    $A \setminus B = \{ x\ |\ x \in A \wedge x \not\in B \} = \{ x\ |\ x \in A \} \cap \{ x\ |\ x \not\in B \} = A \cap B^{c}$
  \end{proof}
\end{st}


\begin{de}
  Het \term{symmetrisch verschil} van twee verzamelingen $A$ en $B$ is de verzameling van alle elementen die in precies \'e\'en van de twee verzamelingen zit.
  \[ A \Delta B = \left\{ x\ |\ (x \in A \wedge x \not\in B) \vee (x \in B \wedge x \not\in A) \right\} \]
\end{de}

\begin{st}
  Het symmetrisch verschil van twee verzamelingen kan worden herschreven als de unie van de twee verschillen.
  \[ A \Delta B = A \nabla B = A \div B = A \ominus B = (A \setminus B) \cup (B \setminus A) \]
\end{st}

\subsubsection{Machtsverzameling}
\label{sec:machtsverzameling}

\begin{de}
  \label{de:machtsverzameling}
  De \term{machtsverzameling} $\mathcal P(A)$ is de \term{verzameling van alle deelverzamelingen} van een verzameling $A$.
  \[ \mathcal P(A) = \left\{ S\ |\ S \subseteq A \right\} \]
\end{de}

\begin{de}
  \label{de:partitie}
  Een partitie $P$ van een verzameling $X$ is een deelverzameling van de machtsverzameling $\mathcal {P}(x)$ van $X$ met de volgende eigenschappen:
  \begin{itemize}
  \item De verzamelingen zijn niet leeg.
    \[ \forall A \in P:\ A \neq \emptyset \]
  \item De verzamelingen zijn onderling disjunct.
    \[ \forall A,B \in P:\ A \neq B \Rightarrow A \cap B = \emptyset \]
  \item De verzamelingen samen vormen $X$.
    \[ \forall x \in X:\ \exists A \in P:\ x \in A \]
  \end{itemize}
\end{de}


\subsection{Koppels en het carthesisch product}
\label{sec:koppels-en-het-carthesisch-product}

\begin{de}
  Een \term{geordend paar} of \term{een koppel} zijn twee elementen die in een bepaalde volgorde samen horen.
  \[ (a,b) \]
\end{de}

\begin{de}
  De \term{gelijkheid} tussen koppels is zo gedifineerd dat de overeenkomstige elementen gelijk zijn.
  \[ (a,b) = (c,d) \Leftrightarrow (a = c \wedge b = c) \] 
\end{de}

\begin{de}
  Het \term{carthesisch product} $A \times B$ van twee verzamelingen $A$ en $B$ is de verzameling der koppels $(x,y)$ met $x \in A$ en $y \in B$
  \[ A \times B = \{ (x,y) \ |\ x \in A \wedge y \in B \} \]
\end{de}

\begin{st}
  Het \term{carthesisch product} is distributief ten opzichte van de unie.
  \[ A \times (B \cup C) = (A \times B) \cup (A \times C) \] 
  \begin{proof}
    \[
    \begin{array}{rll}
    A \times (B \cup C) &= \{ (x,y) \ |\ x \in A \wedge y \in (B \cup C) \}&\\
                        &= \{ (x,y) \ |\ x \in A \wedge (y \in B \vee y \in C) \}&\\
                        &= \{ (x,y) \ |\ (x \in A \wedge y \in B) \vee (x \in A \wedge y \in C) \}&\\
                        &= \{ (x,y) \ |\ (x \in A \wedge y \in B)\} \cup \{ (x,y) \ |\ (x \in A \wedge y \in C) \} &= (A \times B) \cup (A \times C)\\
    \end{array}
    \]
  \end{proof}
\end{st}
\begin{st}
  Het \term{carthesisch product} is distributief ten opzichte van de doorsnede.
  \[ A \times (B \cap C) = (A \times B) \cap (A \times C) \] 
  \begin{proof}
    \[
    \begin{array}{rll}
    A \times (B \cap C) &= \{ (x,y) \ |\ x \in A \wedge y \in (B \cap C) \}&\\
                        &= \{ (x,y) \ |\ x \in A \wedge (y \in B \wedge y \in C) \}&\\
                        &= \{ (x,y) \ |\ (x \in A \wedge y \in B) \wedge (x \in A \wedge y \in C) \}&\\
                        &= \{ (x,y) \ |\ (x \in A \wedge y \in B)\} \cap \{ (x,y) \ |\ (x \in A \wedge y \in C) \} &= (A \times B) \cap (A \times C)\\
    \end{array}
    \]
  \end{proof}
\end{st}

\begin{st}
  Zij $A$, $B$, $C$ en $D$ verzamelingen, dan geldt volgende gelijkheid.
  \[ (A \times B) \cap (C \times D) = (A \cap C) \times (B \cap D) \]
  \begin{proof}
    \[
    \begin{array}{rll}
      (A \times B) \cap (C \times D) &= \{ (x,y) \ |\ x \in A \wedge y \in B \} \cap \{ (x,y) \ |\ x \in C \wedge y \in D \}&\\
                                     &= \{ (x,y) \ |\ x \in A \wedge y \in B \wedge x \in C \wedge y \in D\}&\\
                                     &= \{ (x,y) \ |\ x \in A \wedge x \in C \wedge y \in B \wedge y \in D\}&\\
                                     &= \{ x \ |\ x \in A \wedge x \in C\} \times \{ y \ |\ y \in B \wedge y \in D\} &= (A \cap C) \times (B \cap D)
    \end{array}
    \]
  \end{proof}
\end{st}

\begin{de}
  Het \term{carthesisch product} van een verzameling $A$ met zichzelf wordt wel eens als $A^{2}$ genoteerd.
  \[ A^{2} = A \times A \]
\end{de}

\begin{de}
  Een \term{$n$-koppel} of \term{$n$-tal} zijn $n$ elementen die in een bepaalde volgorde voorkomen.
  \[ (a_{1}, a_{2}, \ldots, a_{n}) \]
\end{de}

\begin{de}
  Het \term{$n$-voudig Carthesis product} tussen $n$ verzamelingen is de verzameling van alle $n$-tallen over die verzamelingen.
  \[ A_{1} \times A_{2} \times \ldots \times A_{n} = \left\{ (a_{1}, a_{2}, \ldots, a_{n}) \ |\ a_{i} \in A_{i} \right\}\]
\end{de}

\begin{de}
  Het \term{$n$-voudig Carthesis product} van een verzameling $A$ met zichzelf wordt als $A^{n}$ genoteerd.
  \[ A^{n} = A \times A \times \ldots \times A\]
\end{de}


\end{document}