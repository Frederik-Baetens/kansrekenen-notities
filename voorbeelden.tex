\documentclass[main.tex]{subfiles}
\begin{document}

\chapter{Voorbeelden}
\label{cha:voorbeelden}



\section{Combinatoriek}
\subsection*{Variaties}
\begin{vb}
  Het aantal manieren om $4$ studenten uit $10$ aan te duiden om $4$ verschillende oefeningen te maken is $V_{10}^{4}$.
\end{vb}

\subsection*{Herhalingsvariaties}
\begin{vb}
  Het aantal verschillende bytes is $\overline{V}_{8}^{2}$.
\end{vb}

\subsection*{Permutaties}
\begin{vb}
  Het aantal manieren om $5$ personen aan een ronde tafel te zetten is $P_{5}$.
\end{vb}

\subsection*{Herhalingspermutaties}
\extra{voorbeeld}

\subsection*{Combinaties}
\begin{vb}  
  Het aantal manieren om een groepje van $4$ studenten uit $10$ aan te duiden is $C^{4}_{10}$.
\end{vb}

\begin{vb}
  Het aantal manieren om twee teams van $6$ uit $12$ spelers te kiezen is $\nicefrac{C^{6}_{12}}{2}$.
\end{vb}

\begin{vb}
  Het aantal manieren om $5$ keer hetzelfde aantal ogen te gooien met $5$ dobbelstenen is $C_{6}^{1}$.
\end{vb}

\begin{vb}
  Het aantal manieren om $4$ keer hetzelfde aantal ogen te gooien met $5$ dobbelstenen is $C_{6}^{1}\cdot C_{5}^{1}$.
\end{vb}

\subsection*{Herhalingscombinaties}
\extra{voorbeeld}

\newpage 
\section{Kansruimten}
\extra{voorbeelden van sigma-algebra}

\begin{vb}
  Het gooien van een dobbelsteen, met mogelijke uitkomsten $\Omega = \{1,2,3,4,5,6\}$ is een stochastisch experiment.
\end{vb}

\begin{vb}
  Bij het gooien van een dobbelsteen is ``het gooien van meer dan drie ogen'' gebeurtenis $A =\{4,5,6\}$.
\end{vb}

\begin{vb}
  Het opgooien van een munstuk, met twee mogelijke uitkomsten $\Omega = \{kop, munt\}$ is een bernoulli experiment
\end{vb}

\begin{vb}
  Twee personen kiezen lukrook een geheel getal tussen $1$ en $5$, maar mogen niet hetzelfde getal kiezen.
  Dit is een stochastisch experiment met universum $\Omega = \{ (a,b) \mid a,b \in \{1,2,3,4,5 \} \wedge a \neq b \}$
\end{vb}

\subsection*{Sigma algebra's}

\begin{vb}
  Voor het experiment van een worp met een muntstuk met universum $\{kop,munt\}$ is $\mathcal{A} = \{ \emptyset, \{kop\}, \{munt\}, \Omega \}$ een $\sigma$-algebra.
\end{vb}

\subsection*{Kansmaten}

\subsection*{Traditionele kansruimten}
\begin{vb}
  In het experiment van het opgooien van een muntstuk is er voor de meetbare ruimte $\{kop,munt\}, \{ \emptyset, \{kop\}, \{munt\}, \Omega \}$ een uniforme kansmaat:
  \[ P(\{kop\}) = P(\{munt\}) = \frac{1}{2} \]
\end{vb}
\extra{dobbelsteen}
\extra{kaartspel}

\extra{geiger-muller teller}



\end{document}

%%% Local Variables:
%%% mode: latex
%%% TeX-master: t
%%% End:
