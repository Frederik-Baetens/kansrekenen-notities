\documentclass[main.tex]{subfiles}
\begin{document}


\chapter{Stochastische Veranderlijken}
\label{cha:stoch-verand}

\section{Inleiding}
\label{sec:inleiding}

\begin{de}
  $\mathcal{B}(\mathbb{R})$ is de $\sigma$-algebra voortgebracht door $\mathcal{C} = \{ ]-\infty,a]\mid -\infty < a < +\infty \}$.
\end{de}

\begin{de}
  Zij $\Omega, \mathcal{A}, P$ een kansruimte.
  Een re\"ele functie $X$ als volgt noemen we een \term{stochastische veranderlijke}, een \term{stochastische variabele}, een \term{stochastiek}, een \term{toevalsvariabele} of een \term{random variabele}.
  \[ X:\ \Omega \rightarrow \mathbb{R} \]
  \[ \forall B\in \mathcal{B}(\mathbb{R}):\ X^{-1}(B) = \{\omega \mid X(\omega) \in B \} \]
  We noemen $X$ dan een $\mathcal{A}$-\term{meetbare afbeelding}.
  
\end{de}

\begin{de}
  Een $\mathcal{B}(\mathbb{R})$-meetbare afbeelding in de meetbare ruimte $\mathbb{R},\mathcal{B}(\mathbb{R})$ noemen we \term{Borel-meetbaar}.
\end{de}

\begin{st}
  Een afbeelding $X: \mathbb{R} \rightarrow \mathbb{R}$ is een stochastische variabele in de meetbare ruimte $\mathbb{R},\mathcal{B}(\mathbb{R})$ als en slechts als het volgende geldt:
  \[ \forall a \in \mathbb{R}:\ X^{-1}(]-\infty,a]) = \{ \omega\mid X(\omega) \le a\} \in \mathcal{A} \]
  \zb
\end{st}

\begin{st}
  Een Borel-meetbare afbeelding introduceert een \term{kansmaat} $P_{X}(B)$ op $\mathcal{B}(\mathbb{R})$.
  \[ P_{X}(B) = P(X \in B) = P(X^{-1}(B)) \]
  \[ P_{X}(B) = P(\{\omega \in \Omega \mid X(\omega) \in B \}) \]
\extra{bewijs}
\end{st}

\begin{de}
  Zij $\Omega,\mathcal{A},P$ een kansruimte en $X: \Omega \rightarrow \mathbb{R}$ een stochastische veranderlijke, dan definieren we de overeenkomstige \term{verdelingsfunctie} (of \term{cumulatieve distributiefunctie}) $F_{X}$ als volgt:
  \[ F: \mathbb{R} \rightarrow \mathbb{R}:\ F_{X}(a) = P_{X}(]-\infty,a]) = P(\{\omega\mid X(\omega) \le a\}) = P(X \le a) \]
\end{de}

\begin{st}
  Zij $X$ een stochastische veranderlijke in de kansruimte $\Omega,\mathcal{A},P$, dan is de afbeelding $F_{X}$ een verdelingsfunctie als en slechts als de volgende beweringen gelden.
  \begin{enumerate}
  \item $F_{X}$ is monotoon stijgend.
    \[ \forall a,b\in \mathbb{R}:\ a\le b \Rightarrow F_{X}(a) \le F_{X}(b) \]
  \item $\lim_{a\rightarrow +\infty}F_{X}(a) = 1$ en $\lim_{a \rightarrow -\infty}F_{X}(a) = 0$
  \item $F_{X}$ is rechts continu.
    \[ \forall a\in\mathbb{R}:\ \lim_{h \overset{>}{\rightarrow} 0}F_{X}(a+h) = F_{X}(a) \]
  \end{enumerate}
\zb
\end{st}

\begin{st}
  Zij $F_{X}$ een verdelingsfunctie voor een kansruimte $\Omega,\mathcal{A},P)$.
  \[ P(a < X \le b) = F_{X}(b) - F_{X}(a) \]
\extra{bewijs}
\end{st}


\begin{st}
  Zij $F_{X}$ een verdelingsfunctie voor een kansruimte $\Omega,\mathcal{A},P)$.
  \[ P(X > a) = 1-F_{X}(a) \]
\extra{bewijs}
\end{st}

\begin{de}
  De \term{kwantielfunctie} $Q_{X}$ voor een kansruimte $\Omega,\mathcal{A},P)$ is de inverse functie van de verdelingsfunctie $F_{X}$.
  De waarde $Q_{X}(p)$ is de kleinste waarde $a$ waarvoor $F_{X}(a) \ge p$ geldt.
\end{de}

\begin{de}
  Het $25\%$, $50\%$ en $75\%$ kwantiel worden respectievelijk het eerste, tweede en derde \term{kwartiel} genoemd.
\end{de}

\begin{de}
  Het tweede kwartiel wordt ook wel de \term{mediaan} genoemd.
\end{de}

\section{Types stochastische veranderlijken}
\label{sec:types-stoch-verand}

\subsection{Discrete stochastische veranderlijken}
\label{sec:discr-stoch-verand}

\begin{de}
  We noemen een stochastische veranderlijke $X$ in een kansruimte $\Omega,\mathcal{A},P$ \term{discreet} als het beeld $p_{i}$ van $X$ slechts in een aftelbaar aantal punten $x_{i}$ niet nul is.
  \[ p_{i} = P(\{\omega \in \Omega \mid X(\omega) = x_{i}\}) = P(X = x_{i}) \]
\end{de}

\begin{de}
  We noemen de rij $(p_{i})_{i}$ de \term{discrete verdeling} van een discrete stochastische veranderlijke $X$ als het volgende geldt:
  \[ \forall i:\ p_{i} \ge 0 \quad\text{ en }\quad \sum_{i}p_{i} = 1 \]
\end{de}

\begin{st}
  De kansverdeling van een discrete stochastische veranderlijke $X$ heeft een eenvoudigere formule:
  \[ F_{X}(a) = P(X \le a) = \sum_{x_{i}\le a}p_{i} \]
\extra{bewijs}
\end{st}

\subsection{Continue stochastische veranderlijken}
\label{sec:cont-stoch-verand}

\begin{de}
  We noemen een stochastische veranderlijke $X$ in een kansruimte $\Omega,\mathcal{A},P$ \term{continu} als de kans op elk punt nul is...
  \[ \forall x\in \Omega: P_{X}(\{x\}) = 0 \]
  ... en de verdelingsfunctie $F_{X}$ een continue functie is.
\end{de}

\begin{de}
  Wanneer de verdelingdsfunctie $F_{X}$ van een continu\"e stochastische veranderlijke continu afleidbaar is, defini\"eren we de \term{dichtheidsfunctie} of \term{kansdichtheid} $f_{X}$ als volgt:
  \[ f_{X}(x):\ \mathbb{R} \rightarrow \mathbb{R}:\ x \rightarrow f_{X}(x) = \frac{dF_{X}(x)}{dx} \]
\end{de}

\begin{st}
  De kansverdeling van een continu\"e stochastische veranderlijke $X$ valt te berekenen met een integraal:
  \[ F_{X}(a) = P(X \le a) = \int_{-\infty}^{a}f_{X}(x)dx \]
\extra{bewijs}
\end{st}

\begin{st}
  Zij $F_{X}$ de kansverdeling van een continu\"e stochastische veranderlijke $X$.
  \[ F_{X}(b) - F_{X}(a) = P(a<X\le b) = \int_{a}^{b}f_{X}(x)dx \]
\extra{bewijs}
\end{st}

\begin{st}
  Zij $F_{X}$ de kansverdeling van een continu\"e stochastische veranderlijke $X$.
  \[ P_{X}(]a,b[) = P_{X}([a,b[) = P_{X}(]a,b]) = P_{X}([a,b]) \]
\extra{bewijs}
\end{st}

\section{Momenten van een stochastische veranderlijken}
\label{sec:momenten-van-een}

\subsection{Verwachtingswaarde en variantie}
\label{sec:verw-en-vari}

\begin{de}
  Zij $\Omega,\mathcal{A},P$ een kansruimte, $X$ een stochastische veranderlijke met verdelingsfunctie $F_{X}$, dan definiert men de \term{verwachtingswaarde} van $X$ als $E[X]$:
  \[
  E[X]
  = \sum_{-\infty}^{+\infty}xdF(x)(x)
  =
  \left\{
    \begin{array}{c l}
      \sum_{i}x_{i}p_{i} &\text{ als } F_{X} \text{ discreet is} \\
      \int_{-\infty}^{+\infty}xf_{X}(x)dx &\text{ als } F_{X} \text{ continu is} 
    \end{array}
  \right.
  \]
\end{de}

\begin{opm}
  Merk op dat $E[X]$ niet steeds bestaat.
\end{opm}

\begin{st}
  Beschouw een Borel-meetbare afbeelding $g:\ \mathbb{R} \rightarrow \mathbb{R}$, dan is $g \circ X$ opnieuw een stochastische verabele.
  \extra{bewijs}
\end{st}

\begin{ei}
  Zij $X: \Omega \rightarrow \mathbb{R}$ een stochastische variabele en $a\in \mathbb{R}$ een constante.
  \[ E[aX] = aE[X] \]
\extra{bewijs}
\end{ei}

\begin{ei}
  Zij $X: \Omega \rightarrow \mathbb{R}$ een stochastische variabele en $b\in \mathbb{R}$ een constante.
  \[ E[X+b] = E[X] + b \]
\extra{bewijs}
\end{ei}

\begin{ei}
  Zij $X: \Omega \rightarrow \mathbb{R}$ een stochastische variabele en $b\in \mathbb{R}$ een constante.
  \[ E[b] = b \]
\extra{bewijs}
\end{ei}

\begin{ei}
  \[ |E[X]| \le E[|X|] \]
\extra{bewijs}
\end{ei}

\begin{de}
  De verwachtingswaarde $E[X]$ wordt ook wel het \term{eerste moment} genoemd.
\end{de}

\begin{de}
  Het \term{eerste absolute moment} definieren we als $E[|X|]$.
\end{de}

\begin{de}
  De \term{variantie} van een stochastische veranderlijke $X$ definieren we als $Var[X]$:
  \[ Var[X] = E[(X-E[X])^{2}] \]
\end{de}

\begin{de}
  De \term{standaardafwijking} van een stochastische veranderlijke $X$ definieren we als $\sqrt{Var[X]}$.
\end{de}

\begin{st}
  We kunnen $Var[X]$ eenvoudig berekenen.
  \[
  Var[X]
  =
  \left\{
    \begin{array}{c l}
      \sum_{i}(x_{i}-E[X])^{2}p_{i} &\text{ als } F_{X} \text{ discreet is} \\
      \int_{-\infty}^{+\infty}(x-E[X])^{2}f_{X}(x)dx &\text{ als } F_{X} \text{ continu is} 
    \end{array}
  \right.
  \]
\extra{bewijs}
\end{st}

\begin{ei}
  Zij $X: \Omega \rightarrow \mathbb{R}$ een stochastische variabele en $a\in \mathbb{R}$ een constante.
  \[ Var[aX] = a^{2}Var[X] \]

  \begin{proof}
    \begin{align*}
      Var[aX]
      &= E\left[\left(aX - E[aX]\right)^{2}\right]\\
      &= E\left[a^{2}X^{2} + E[aX]^{2} - 2aXE[aX]\right]\\
      &= E\left[a^{2}\left(X^{2}+E[X]^{2} - 2XE[X]\right)\right]\\
      &= a^{2}E\left[\left(X - E[X]\right)^{2}\right]\\
      &= a^{2}Var[X]
    \end{align*}
  \end{proof}
\end{ei}

\begin{ei}
  Zij $X: \Omega \rightarrow \mathbb{R}$ een stochastische variabele en $b\in \mathbb{R}$ een constante.
  \[ Var[X+b] = Var[X] \]

  \begin{proof}
    \begin{align*}
      Var[X+b]
      &= E\left[\left((X+b) - E[X+b]\right)^{2}\right]\\
      &= E\left[(X+b)^{2} + E[X+b]^{2} -2(X+b)E[X+b]\right]\\
      &= E\left[X^{2} + 2bX+b^{2} + (E[X]+b)^{2} - 2(X+b)(E[X]+b)\right]\\
      &= E\left[X^{2} + 2bX+b^{2} + E[X]^{2} + 2bE[X] + b^{2} - 2XE[X] -2Xb -2bE[X] -2b^{2}\right]\\
      &= E\left[X^{2} + \cancel{2bX} + \cancel{b^{2}} + E[X]^{2} + \cancel{2bE[X]} + \cancel{b^{2}} - 2XE[X] -\cancel{2bX} -\cancel{2bE[X]} - \cancel{2b^{2}}\right]\\
      &= E\left[X^{2} + E[X]^{2} - 2XE[X]\right]\\
      &= E\left[\left(X - E[X]\right)^{2}\right]\\
      &= Var[X]
    \end{align*}
  \end{proof}
\end{ei}

\begin{ei}
  Zij $X: \Omega \rightarrow \mathbb{R}$ een stochastische variabele en $b\in \mathbb{R}$ een constante.
  \[ Var[b] = 0 \]

  \begin{proof}
    Dit volgt meteen uit $E[b] = 0$.
    \begin{align*}
      Var[b]
      = E\left[\left(b - E[b]\right)^{2}\right]
      = E\left[b^{2}\right]
      = 0
    \end{align*}
  \end{proof}
\end{ei}

\subsection{Som en product van stochastische veranderlijken}
\label{sec:som-en-product}

\begin{de}
  Zij $X$ en $Y$ beide stochastische variabelen in een kansruimte $\Omega,\mathcal{A},P$, dan definieren we de \term{som van stochastische variabelen} als $X+Y$:
  \[ (X+Y)(\omega) = X(\omega) + Y(\omega) \]
\end{de}

\begin{st}
  De som van twee stochastische variabelen is opnieuw een stochastische variabele.
\extra{bewijs}
\end{st}

\begin{de}
  Zij $X$ en $Y$ beide stochastische variabelen in een kansruimte $\Omega,\mathcal{A},P$, dan definieren we het \term{product van stochastische variabelen} als $XY$:
  \[ (X+Y)(\omega) = X(\omega)Y(\omega) \]
\end{de}

\begin{st}
  Het product van twee stochastische variabelen is opnieuw een stochastische variabele.
\extra{bewijs}
\end{st}

\begin{ei}
  Zij $X$ en $Y$ beide stochastische variabelen in een kansruimte $\Omega,\mathcal{A},P$.
  \[ E[X+Y] = E[X] + E[Y] \]

  \begin{proof}
    We bewijzen dit enkel in het geval dat $\Omega = \{\omega_{1},\omega_{2},\dotsc\}$ aftelbaar is.
    \begin{align*}
      E[X+Y]
      &= \sum_{i}(X+Y)(\omega_{i})P(\{\omega_{i}\})\\
      &= \sum_{i}X(\omega_{i})P(\{\omega_{i}\}) + \sum_{i}Y(\omega_{i})P(\{\omega_{i}\})\\
      &= E[X] + E[Y]
    \end{align*}
  \end{proof}
\end{ei}

\begin{de}
  Twee stochastische variabelen $X$ en $Y$ in een kansruimte $\Omega,\mathcal{A},P$ noemen we \term{onafhankelijk} als het volgende geldt:
  \[ P((X\in A)\cap (Y\in B)) = P(X\in A)P(Y\in A) \]
\end{de}

\begin{ei}
  Zij $X$ en $Y$ twee \textbf{onafhankelijke} stochastische variabelen in een kansruimte $\Omega,\mathcal{A},P$.
  \[ E[XY] = E[X] E[Y] \]

  \begin{proof}
    Gevalsonderscheid:
    \begin{itemize}
    \item $X$ en $Y$ zijn discreet.
      \begin{align*}
        E[XY]
        &= \sum_{j}\sum_{k}x_{j}x_{k}P(X=x_{j},Y=y_{k})\\
        &= \sum_{j}\sum_{k}x_{j}x_{k}P(X=x_{j})P(Y=y_{k})\\
        &= \left( \sum_{j}x_{j}P(X=x_{j}) \right)\left( \sum_{k}x_{k}P(Y=y_{k}) \right)
        &= E[X]E[Y]
      \end{align*}
      \extra{bewijs de andere gevallen}
    \end{itemize}
  \end{proof}
\end{ei}

\begin{ei}
  Zij $X$ en $Y$ twee \textbf{onafhankelijke} stochastische variabelen in een kansruimte $\Omega,\mathcal{A},P$.
  \[ Var[X+Y] = Var[X] + Var[Y] \]

  \begin{proof}
    \begin{align*}
      Var[X+Y]
      &= E\left[\left((X + Y) - E[X+Y]\right)^{2}\right]\\
      &= E\left[\left((X + Y) - (E[X] + E[Y])\right)^{2}\right]\\
      &= E\left[\left((X - E[X]) + (Y - E[Y])\right)^{2}\right]\\
      &= E\left[(X - E[X])^{2} + (Y - E[Y])^{2} + 2(X - E[X])(Y - E[Y])\right]\\
      &= E\left[(X - E[X])^{2}\right] + E\left[(Y - E[Y])^{2}\right] + E\left[2(X - E[X])(Y - E[Y])\right]\\
      &= Var[X] + Var[Y] + 2E\left[X-E[X]\right]E\left[Y-E[Y]\right]\\
      &= Var[X] + Var[Y] + 2(E[X] - E[X])(E[Y]-E[Y])\\
      &= Var[X] + Var[Y]
    \end{align*}
  \end{proof}
\end{ei}

\begin{ei}
  Zij $X$ en $Y$ twee \textbf{onafhankelijke} stochastische variabelen in een kansruimte $\Omega,\mathcal{A},P$.
  \[ Vay[XY] = Var[X]Var[Y] + E[X]^{2}Var[Y] + Var[X]E[Y]^{2} \ge Var[X]Var[Y] \]

\extra{bewijs}
\end{ei}

\begin{st}
  \[ Var[X] = E[X^{2}] - E[X]^{2} \]

  \begin{proof}
    \begin{align*}
      Var[X]
      &= E\left[(X-E[X])^{2}\right]\\
      &= E\left[X^{2}+E[X]^{2} - 2XE[X]\right]\\
      &= E\left[X^{2}\right] E\left[E[X]^{2}\right] -2E\left[X E[X]\right]\\
      &= E\left[X^{2}\right] + E[X]^{2} - 2E[X]E[X]\\
      &= E\left[X^{2}\right] - E[X]^{2}
    \end{align*}
  \end{proof}
\end{st}

\subsection{Ongelijkheden}
\label{sec:ongelijkheden}

\subsubsection{Grenzen voor het eerste absolute moment}
\label{sec:grenzen-voor-het}

\begin{st}
  Zij $\Omega,\mathcal{A},P$ een kansruimte en $X$ een willekeurige stochastische variabele op $\Omega$.
  Als $E[|X|]$ bestaat, dan bestaat ook $E[X]$.
\extra{bewijs} 
\end{st}

\begin{st}
  Zij $\Omega,\mathcal{A},P$ een kansruimte en $X$ een willekeurige stochastische variabele op $\Omega$.
  \[ E[|X|] \in \interval{\sum_{n=1}^{\infty}P(|X| \ge n)}{1 + \sum_{n=1}^{\infty}P(|X| \ge n)} \]

  \begin{proof}
    Gevalsonderscheid:
    \begin{itemize}
    \item $X$ is een discrete stochastische variabele
\extra{bewijs}
    \item $X$ is een continue stochastische variabele:\\
      Beschouw de volgende partitie van $\mathbb{R}$:
      \[ \mathbb{R} = \interval[open]{-\infty}{\infty} = \bigcup_{n=0}^{\infty}\left\{ x \in\mathbb{R} \mid |x| \in \interval[open right]{n}{n+1} \right\} \]
      Benoem hierin $A_{n}= \left\{ x \in\mathbb{R} \mid |x| \in \interval[open right]{n}{n+1} \right\}$.
      Merk op dat we $A_{n}$ dan kunnen herschrijven als volgt:
      \[ A_{n} = \interval[open left]{-(n+1)}{-n} \cup \interval[open right]{n}{n+1} \]
      We herschrijven $E[|X|]$ aan de hand van deze partitie:
      \begin{align}
        E[|X|]
        = \int_{\mathbb{R}}|x|f_{X}(x)\ dx
        = \sum_{n}\int_{x\in A_{n}}|x|f_{X}(x)\ dx'
      \end{align}
      Als $x$ tot een bepaalde $A_{n}$ behoort, geldt $|x| \in \interval[open right]{n}{n+1}$ en dus het volgende: (vermenigvuldig beide kanten met $\1_{A_{n}}(x)$.
      \[ |x| \1_{A_{n}}(x) \in \interval{n\1_{A_{n}}(x)}{(n+1)\1_{A_{n}}(x)} \]
      Hieruit volgt meteen het volgende (someer en integreer beide kanten):
      \[
      \sum_{n=0}^{\infty}\int_{\mathbb{R}}|x|\1_{A_{n}}(x)f_{X}(x)\ dx
      \in
      \interval
      {\sum_{n=0}^{\infty}\int_{\mathbb{R}}n\1_{A_{n}}(x)f_{X}(x)\ dx}
      {\sum_{n=0}^{\infty}\int_{\mathbb{R}}(n+1)\1_{A_{n}}(x)f_{X}(x)\ dx}
      \]
      We weten het volgende:
      \[ \sum_{n}\int_{\mathbb{R}}\1_{A_{n}}(x)f_{X}(x) = \sum_{n}P(X\in A_{n}) = P(X\in\mathbb{R}) = 1 \]
      Hieruit volgt meteen dit:
      \[ E[|X|] \in \interval{\sum_{n}nP(X\in A_{n})}{\sum_{n}nP(X\in A_{n}) + 1} \]
      Er rest ons nog aan te tonen dat $\sum_{n}nP(X\in A_{n})$ gelijk is aan $sum_{n}P(|X| \ge n)$:
      \begin{align*}
        \sum_{n}P(|X| \ge n)
        &= \sum_{n}P\left(\bigcup_{m=n}^{\infty}\left\{ |X| \in \interval{m}{m+1} \right\}\right)\\
        &= \sum_{n}\sum_{m=n}P(\left\{ |X| \in \interval{m}{m+1} \right\})\\
        &= \sum_{n}\sum_{m=n}P(X\in A_{m})\\
        &= \sum_{m}\sum_{n=1}^{m}P(X\in A_{m})\\
        &= \sum_{m}mP(X\in A_{m})
      \end{align*}
      Merk op dat die tweede gelijkheid enkel geldt omdat de $A_{n}$ een partitie vormen.
      In de vierde gelijkheid is de wet van Fubini toegepast.
    \end{itemize}
  \end{proof}
\end{st}

\begin{gev}
  $E(|X|)$ bestaat als en slechts als $\sum_{n=1}^{\infty}P(|X| \ge n)$ convergeert.
  \extra{bewijs: werk uit}
\end{gev}

\begin{st}
  Zij $X$ een stochastische variabele, dan bestaat $E[|X|^{k}]$ voor alle $k$ kleiner dan $n$ als $E[|X|^{n}]$ bestaat.
\extra{bewijs}
\end{st}

\begin{gev}
  Zij $X$ een positieve stochastische veranderlijke die alleen gehele waarden kan aannemen.
  \[ E[X] = \sum_{n=1}^{\infty}P(X\ge n) \]
\extra{bewijs}
\end{gev}

\subsubsection{De ongelijkheid van Chebyshev}
\label{sec:de-ongelijkheid-van}

\begin{st}
  De \term{ongelijkheid van Chebyshev}\\
  Zij $X$ een stochastische variabele en $\phi:\mathbb{R} \rightarrow \mathbb{R}$ een functe zodat $E[\phi(X)]$ bestaat.
  \begin{enumerate}
  \item Als $\phi$ positief, even en niet-dalend is voor alle $x\ge 0$ geldt het volgende:
    \[ \forall a \ge 0:\ P(|X| \ge a) \le \frac{E[\phi(X)]}{\phi(a)} \]
  \item Als $\phi$ positief en niet-dalend is voor alle $x\in\mathbb{R}$ geldt het volgende:
    \[ \forall a \in \mathbb{R}:\ P(X \ge a) \le \frac{E[\phi(X)]}{\phi(a)} \]
  \end{enumerate}

  \begin{proof}
    Gevalsonderscheid:
    \begin{itemize}
    \item $X$ is een discrete stochastische variabele.
      \extra{bewijs afmaken}
    \item $X$ is een continue stochastische variabele. 
      We berekenen eenvoudigweg de verwachtingswaarde:
      \begin{enumerate}
      \item 
        \begin{align*}
          E[\phi(X)]
          &= \int_{\mathbb{R}}\phi(x)f_{X}(x)\ dx\\
          &= \int_{-\infty}^{a}\phi(x)f_{X}(x)\ dx + \int_{-a}^{a}\phi(x)f_{X}(x)\ dx + \int_{a}^{+\infty}\phi(x)f_{X}(x)\ dx\\
          &\ge \int_{-\infty}^{a}\phi(x)f_{X}(x)\ dx + \int_{a}^{+\infty}\phi(x)f_{X}(x)\ dx\\
          &\ge \phi(-a)\int_{-\infty}^{a}f_{X}(x)\ dx + \phi(a)\int_{a}^{+\infty}\phi(x)f_{X}(x)\ dx\\
          &= \phi(a) \left( \int_{-\infty}^{a}f_{X}(x)\ dx + \int_{a}^{+\infty}\phi(x)f_{X}(x)\ dx\right)\\
          &= \phi(a) \left(P(X \le a) + P(X \ge a)\right)\\
          &= \phi(a)P(|X| \ge a)
        \end{align*}
        Merk op dat de derde gelijkheid enkel geldt omdat $\phi$ even is.
        De tweede ongelijkheid geldt omad
        Hieruit volgt rechtstreeks de stelling.
      \item
        \begin{align*}
          E[\phi(X)]
          &= \int_{\mathbb{R}}\phi(x)f_{X}(x)\ dx\\
          &= \int_{-\infty}^{a}\phi(x)f_{X}(x)\ dx + \int_{-a}^{a}\phi(x)f_{X}(x)\ dx + \int_{a}^{+\infty}\phi(x)f_{X}(x)\ dx\\
          &\ge \int_{a}^{+\infty}\phi(x)f_{X}(x)\ dx\\
          &\ge \phi(a)\int_{a}^{+\infty}f_{X}(x)\ dx\\
          &= \phi(a)P(X \ge a)
        \end{align*}
        Hieruit volgt rechtstreeks de stelling.        
      \end{enumerate}
    \end{itemize}
  \end{proof}
\end{st}

\begin{gev}
  Zij $X$ een stochastische variabele met $E[|X|^{n}] < \infty, n>0$, dan geldt het volgende:
  \[ \forall a> 0:\ P(|X| \ge a) \le a^{-n}E[|X|^{n}] \]
\end{gev}
\begin{gev}
  Zij $X$ een stochastische variabele met $E[|X|^{n}] < \infty, n>0$, $E[X] = \mu$ en $Var[X] = sigma^{2} < \infty$, dan geldt:
  \[ \forall a > 0:\ P(|X-\mu| \ge a) \le \frac{\sigma^{2}}{a^{2}} \]
\end{gev}

\subsection{Hogere momenten en momentgenererende functie}
\label{sec:hogere-momenten-en}

\begin{de}
  Voor elk natuurlijk getal $k\ge 1$ definieert men het \term{ruwe moment} van orde $k$ als $\alpha_{k}(X)$...
  \[ \alpha_{k}(X) =E[X^{k}] \]
  ... en het \term{centrale moment} van orde $k$ als $\mu_{k}(X)$.
  \[ \mu_{k}(X) = E[(X-E[X])^{k}] \]
\end{de}

\begin{opm}
  Het eerste ruwe moment is dus het gemiddelde en het tweede centrale moment is de variantie.
\end{opm}

\begin{de}
  De \term{momentgenererende functie} (MGF) van $X$ is gedefinieerd als $M_{X}(t)$.
  \[
  M_{X}:\ \mathbb{R} \rightarrow \mathbb{R}^{+}_{0}:\ t \rightarrow M_{X}(t) = E[e^{tX}]
  = \left\{
    \begin{array}{cl}
      \sum_{i}e^{tx}p_{i} & \text{ als } F_{X} \text{ discreet is}\\
      \int_{-\infty}^{+\infty}e^{tx}f(x)dx & \text{ als } F_{X} \text{ continu is}
    \end{array}
    \right.
  \]
\end{de}

\begin{st}
  \[ M_{X}(t) = \sum_{i=1}^{\infty} \frac{\alpha_{i}t^{i}}{i!} \]

  \begin{proof}
    We gebruiken de Taylorontwikkeling van $e^{tX}$ rond $0$:
    \[ e^{tX} = \sum_{k}\frac{tX^{X}}{k!} \]
    We nemen nu van beide leden de verwachtingswaarde:
    \[
    E\left[e^{tX}\right]
    = E\left[\sum_{k}\frac{X^{tk}}{k!}\right]
    = \sum_{k}\frac{t^{k}}{k!}E\left[X^{k}\right]
    = \sum_{i=1} \frac{\alpha_{i}t^{i}}{i!}
    \]
  \end{proof}
\end{st}

\begin{gev}
  \[ \alpha_{k} = \left. \frac{d^{k}}{dt^{k}}M_{X}(t) \right|_{t=0} \]
\extra{werk uit}
\end{gev}

\begin{st}
  Twee kansmaten zijn gelijk als en slechts hun MGF gelijk zijn.
\extra{bewijs}
\end{st}

\begin{st}
  Zij $X$ en $Y$ twee stochastische veranderlijken zodat $Y=aX+b$.
  \[ M_{Y}(t) = e^{bt}M_{X}(at) \]

  \begin{proof}
    \begin{align*}
      M_{Y}(t) = E\left[e^{tY}\right] = E\left[e^{t(aX+b)}\right] = E\left[e^{taX}e^{tb}\right] = e^{tb}E\left[e^{taX}\right] = e^{bt}M_{X}(at)
    \end{align*}
  \end{proof}
\end{st}

\begin{ei}
  Zij $X$ en $Y$ onafhankelijke stochastische veranderlijken, dan geldt het volgende:
  \[ \forall t\in\mathbb{R}: M_{X+Y}(t) = M_{X}(t)M_{Y}(t) \]

  \begin{proof}
    Kies $t\in\mathbb{R}$ willekeurig.
    \begin{align*}
      M_{X+Y}(t) = E\left[e^{t(X+Y)}\right] = E\left[e^{tX}e^{tY}\right] = E\left[e^{tX}\right]E\left[e^{tY}\right] = M_{X}(t)M_{Y}(t)
    \end{align*}
  \end{proof}
\end{ei}


\section{Kentallen}
\label{sec:kentallen}


\subsection{Kentallen van locatie}
\label{sec:kent-van-locat}

\subsubsection{Gemiddelde}
\label{sec:gemiddelde}

\begin{de}
  We definieren het \term{gemiddelde} $\mu(X)$ van een stochastische variabele $X$ als $E[X]$.
\end{de}

\begin{figure}[H]
  \centering
  \begin{tabular}{|c|c|}
    \hline
    Voordelen & Nadelen\\
    \hline \hline
    Eenvoudig & Niet robuust\\
    Expliciet & Hoeft niet te bestaan\\
    Lineair &\\
    \hline
  \end{tabular}
\end{figure}

\begin{st}
  Als een verdeling $f$ symmetrisch is ten opzichte van een middelpunt $c$...:
  \[ \forall t:\ f(c-t) = f(c+t) \]
  ... dan is $c$ zowel het gemiddelde van de bijhorende stochastische variabele als het gemiddelde bestaat.
\extra{bewijs}
\end{st}


\subsubsection{Mediaan}
\label{sec:mediaan}

\begin{de}
  We definieren de \term{mediaan} $Med(X)$ van een stochastische variabele $X$ als het tweede kwartiel:
  \begin{enumerate}
  \item $F^{-1}(\nicefrac{1}{2})$ als er precies \'e\'en $x$ bestaat zodat $F(x) = \nicefrac{1}{2}$ geldt.
  \item Het punt waar $F$ een discontinu\"e sprong maakt van $<\nicefrac{1}{2}$ naar $>\nicefrac{1}{2}$ als er g\'e\'en zo'n $x$-en bestaan
  \item Het middelpunt van het interval waarop $F(x) = \nicefrac{1}{2}$ geldt als er meerdere van zo'n $x$-en bestaan.
  \end{enumerate}
\end{de}

\begin{opm}
  Via dezelfde regels kunnen we de kwantielfunctie definieren zodat elk beeld ervan bestaat.
\end{opm}

\begin{figure}[H]
  \centering
  \begin{tabular}{|c|c|}
    \hline
    Voordelen & Nadelen\\
    \hline \hline
    Robuuster & Niet altijd eenvoudig te berekenen\\
    Bestaat altijd  &\\
    \hline
  \end{tabular}
\end{figure}

\begin{st}
  Als een verdeling $f$ symmetrisch is ten opzichte van een middelpunt $c$...:
  \[ \forall t:\ f(c-t) = f(c+t) \]
  ... dan is $c$ de mediaan van de bijhorende stochastische variabele.
\extra{bewijs}
\end{st}

\begin{st}
  Zij $g$ een monotone re\"ele functie en $X$ een stochastische variabele:
  \[ Med(g(X)) = g(Med(x)) \]
\extra{bewijs}
\end{st}

\subsubsection{Modus}
\label{sec:modus}

\begin{de}
  De \term{modus} van een stochastische variabele wordt gedefinieerd als de meest voorkomende waarde.
\end{de}

\begin{figure}[H]
  \centering
  \begin{tabular}{|c|c|}
    \hline
    Voordelen & Nadelen\\
    \hline \hline
     & Niet altijd uniek\\
    \hline
  \end{tabular}
\end{figure}

\begin{de}
  Een verdeling met twee modi noemt men een \term{bimodiale verdeling}.
\end{de}

\begin{de}
  Een verdeling met meer dan twee modi noemen we een \term{multimodiale verdeling}.
\end{de}

\subsection{Kentallen van schaal}
\label{sec:kentallen-van-schaal}

\subsubsection{Variantie en standaardafwijking}
\label{sec:vari-en-stand}

\begin{de}
  De \term{variantie} van een stochastische veranderlijke $X$ definieren we als $Var[X]$ met notatie $\sigma(X)^{2}$.
\end{de}

\begin{de}
  De \term{standaardafwijking} van een stochastische veranderlijke $X$ definieren we als $\sigma = \sqrt{Var[X]}$.
\end{de}

\begin{figure}[H]
  \centering
  \begin{tabular}{|c|c|}
    \hline
    Voordelen & Nadelen\\
    \hline \hline
    Makkelijk te berekenen & Hoeft niet te bestaan\\
    Mooie eigenschappen & Niet robuust\\
    \hline
  \end{tabular}
\end{figure}

\subsubsection{Interkwartiel}
\label{sec:interkwartiel}

\begin{de}
  Het \term{interkwartiel} (\term{IQR}) of de \term{interkwartielafstand} is gedefinieerd als de afstand tussen het eerste en het derde kwartiel.
  \[ IQR = F^{-1}(\nicefrac{3}{4}) = F^{-1}(\nicefrac{1}{4}) \]
\end{de}

\begin{figure}[H]
  \centering
  \begin{tabular}{|c|c|}
    \hline
    Voordelen & Nadelen\\
    \hline \hline
    Bestaat altijd & Bezit niet veel informatie\\
    Robuust & \\
    \hline
  \end{tabular}
\end{figure}

\subsubsection{Median absolute deviation}
\label{sec:medi-absol-devi}

\begin{de}
  De \term{median absolute deviation} (\term{MAD}) is gegeven door de mediane absolute afwijking van de mediaan.
  \[ MAD = Med|X - Med(X)| \]
\end{de}

\begin{figure}[H]
  \centering
  \begin{tabular}{|c|c|}
    \hline
    Voordelen & Nadelen\\
    \hline \hline
    Bestaat altijd & Niet even eenvoudig\\
    Nog robuuster & Bezit niet veel informatie\\
    \hline
  \end{tabular}
\end{figure}

\begin{st}
  Voor symmetrische verdelingen is het interkwartiel precies gelijk aan twee keer de MAD.
\end{st}

\subsubsection{Variantiebreedte}
\label{sec:variantiebreedte}

\begin{de}
  De \term{variantiebreedte} (\term{range}) is de totale breedte waarover de kansmassa zich uitstrekt.
\end{de}

\begin{figure}[H]
  \centering
  \begin{tabular}{|c|c|}
    \hline
    Voordelen & Nadelen\\
    \hline \hline
    & Meestal $+\infty$ \\
    & Niet robuust\\
    \hline
  \end{tabular}
\end{figure}

\subsubsection{Variatieco\"efficient}
\label{sec:variatiecoefficient}

\begin{de}
  De \term{variatieco\"efficient} $\nu(X)$ van een stochastische variabele $X$ vergelijkt de standaardafwijking met het gemiddelde.
  \[ \nu(X) = \frac{\sqrt{Var[X]}}{E[X]} \]
\end{de}

\extra{voordelen en nadelen}

\subsection{Kentallen van scheefheid}
\label{sec:kent-van-sche}

\subsubsection{Derde contrale moment}

\begin{de}
  Het \term{derde centrale moment} $\mu_{3}$ wordt nul wanneer de verdeling van $X$ symmetrisch is.
  \[ \mu_{3}(X) = E\left[\left(X-E[X]\right)^{2}\right] \]
\end{de}

\subsubsection{Scheefheidscoefficient}

\begin{de}
  De \term{scheefheidsco\"efficient} van een stochastische variable definieren we als $\gamma_{1}$.
  \[ \gamma_{1} = \frac{\mu_{3}}{\sigma^{3}} \]
\end{de}

\extra{voordelen en nadelen}


\section{Belangrijke verdelingen}
\label{sec:belangr-verd}

\subsection{Discrete verdelingen}
\label{sec:discrete-verdelingen}

\subsubsection{Discrete uniforme verdeling}
\label{sec:discr-unif-verd}

\begin{de}
  De \term{discrete uniforme verdeling} is gedefinieerd op een eindig universum $\Omega=\{x_{1},\dotsc,x_{n}\}$.
  \[ \forall i \in \{1,\dotsc,n\}:\ p_{i} = P(\{x_{i}\}) = \nicefrac{1}{n} \]
\end{de}


\subsubsection{Bernoulli verdeling}
\label{sec:bernoulli-verdeling}

\begin{de}
  Een \term{Bernoulli verdeling} is gedefinieerd voor een Bernoulli experiment.
  \[
  \begin{cases}
    P(X=1) &= p\\
    P(X=0) &= q = 1-p
  \end{cases}
  \]
  We noemen $p$ de \term{kans op succes}.
  We noteren een stochastische variabele met een Bernoulli verdeling als volgt:
  \[ X \sim \mathcal{B}(1,p) \]
\end{de}


\subsubsection{Binomiaalverdeleing}
\label{sec:binomiaalverdeleing}

\begin{de}
  Een \term{binomiaalverdeling} is de verdeling van de som $Y$ van $n$ bernoulli-verdeelde stochastische veranderlijken $X_{i}$ met slaagkans $p$.
  \[ Y = \sum_{i=1}^{n}X_{i}\]
  \[ Y \sim \mathcal{B}(n,p) \]
  $\Omega$ is dus $\{1,\dotsc,n\}$.
  \[ P(Y=k) = \binom{n}{k}p^{k}q^{n-k} \]
\end{de}

\subsubsection{Geometrische verdeling}
\label{sec:geom-verd}

\begin{de}
  Een \term{Geometrische verdeling} is de verdeling die optreedt als we een bernoulli experiment met slaagkans $\theta$ uitvoeren tot het slaagt.
  \[ P(X=i) = (1-\theta)^{i}p \]
  \[ X \sim Geom(\theta) \]
\end{de}


\subsubsection{Negatief binomiaalverdeling}
\label{sec:negat-binom}

\begin{de}
  Een \term{Geometrische verdeling} is de verdeling die optreedt als we een bernoulli experiment met slaagkans $p$ uitvoeren tot het $r$ keer slaagt.
  \[ P(X=i) = \binom{i+r-1}{i}(1-\theta)^{r}\theta^{i} \]
  \[ X \sim NB(r,\theta) \]
\end{de}


\subsubsection{Hypergeometrische verdeling}
\label{sec:hyperg-verd}

\begin{de}
  Een \term{hypergeometrische verdeling} is een verdeling $Y$ over een universum $\{1,\dotsc,n\}$ als volgt:
  \[ P(Y = i) = \frac{\binom{s}{i}\binom{N-s}{n-i}}{\binom{N}{n}}\]
  \[ Y \sim \mathcal{H}(N,s,n) \]
  Deze verdeling bekomen we als we $n$ experimienten doen die op $s$ manieren kunnen slagen en $N-s$ manieren kunnen falen, maar wel maar $1$ keer per maniet.
  $Y$ is dan de verdeling van het aantal successen
\end{de}

\subsubsection{Poissonverdeling}
\label{sec:poissonverdeling}

\begin{de}
  De \term{Poissonverdeling} $\mathcal{P}(\alpha)$ met parameter $a\in \mathbb{R}^{+}$ is gedefinieerd op $\Omega = \mathbb{N}$ als volgt:
  \[ p_{i} = P(X=i) = \frac{\alpha^{i}}{i!}e^{-\alpha} \]
  Deze verdeling bekone we als we bekijken hoeveel keer een bepaalde gebeurtenis voorvalt binnen een vast interval.
  $\alpha$ wordt de risicoparameter genoemd.
\end{de}

\subsection{Continue verdelingen}
\label{sec:continue-verdelingen}

\subsubsection{Continue uniforme verdelingen}
\label{sec:cont-unif-verd}

\begin{de}
  De stochastische variabele $X$ volgt een \term{continue uniforme verdeling} op $[a,b]$ als $X$ de volgende dichtheid heeft.
  \[ f_{a,b}(x) = \frac{1}{b-a}\1_{[a,b]}(x) \]
\end{de}

\subsubsection{Exponentiele verdeling}
\label{sec:expon-verd}

\begin{de}
  De \term{exponentiele verdeling} met parameter $a$ is een stochastische veranderlijke met de volgende dichtheidsfunctie.
  \[
  f_{a}(t) =
  \left\{
    \begin{array}{cl}
      ae^{-at} &\text{ als } t \ge 0\\
      0      &\text{ als } t < 0\\
    \end{array}
  \right.
  \]
  \[ X \sim \varepsilon(\alpha) \]
\end{de}

\subsubsection{Univariate normale verdeling}
\label{sec:univ-norm-verd}

\begin{de}
  De \term{standaard normale verdeling} $\mathcal{N}(0,1)$ is gedefinieerd door haar dichtheidsfunctie $\phi(x)$:
  \[ \phi(x) = \frac{e^{-\frac{1}{2}x^{2}}}{\sqrt{2\pi}} \]
\end{de}

\begin{de}
  Een stochastische veranderlijke die standaard normaal verdeeld is wordt genoteerd met $Z$.
\end{de}

\begin{de}
  De verdelingsfunctie van een standaard normale verdeling noteren we als $\Phi(x)$.
  \[ \Phi(x) = \int_{-\infty}^{x}\phi(t)dt \]
\end{de}

\begin{de}
  De \term{algemene normale verdeling} $\mathcal{N}(\mu,\sigma^{2})$ met $\mu \in \mathbb{R}$ en $\sigma \in \mathbb{R}^{+}$ is de verdeling van $X = \sigma Z + \mu$.
  \[ f_{\mu, \sigma} = \frac{e^{-\frac{(x-\mu)^{2}}{2\sigma^{2}}}}{\sigma\sqrt{2}\pi} \]
\end{de}

\begin{st}
  \[ F_{X}(x) = F_{Z} \left( \frac{x-\mu}{\sigma} \right) = \Phi\left( \frac{x-\mu}{\sigma} \right) \]
\extra{bewijs: dit is wel belangrijk.}
\end{st}


\subsubsection{$\chi_{n}^2$-verdeling}
\label{sec:chi_n2-verdeling}

\begin{de}
  De \term{gammafunctie} is als volgt gedefinieerd:
  \[
  \Gamma(t) = \int_{0}^{\infty}x^{t-1}e^{-x}dx \text{ voor } t > 0
  \]
\end{de}

\begin{st}
  \[ \forall t \in \mathbb{R}^{+}: \Gamma(t+1) = t\gamma(t) \]
  \extra{bewijs}
\end{st}

\begin{st}
  \[ \forall n \in \mathbb{N}: \Gamma(n+1) = n! \]
  \extra{bewijs}
\end{st}

\begin{st}
  \[ \Gamma\left(\frac{1}{2}\right) = \sqrt{\pi} \]
  \extra{bewijs}
\end{st}

\begin{st}
  De \term{formule van Stirling}
  \[
  \Gamma(t) \sim \left(\frac{t-1}{e}\right)^{t-1}\sqrt{2\pi(t-1)}
  \]
  \[
  \lim_{t\rightarrow \infty}\frac{\Gamma(t)}{\left(\frac{t-1}{e}\right)^{t-1}\sqrt{2\pi(t-1)}} = 1
  \]
 \extra{bewijs: waarom staat dit hier eigenlijk? gebruiken we het ooit?}
\end{st}

\begin{de}
  Zij $Z_{1}, \dotsc, Z_{n}$ $n$ standaard normale verdelingen.
  De som $X$ van deze $Z_{i}$ noemen we \term{chi-kwadraat} verdeeld met $n$ vrijheidsgraden.
  \[ X \sim \chi_{n}^{2} \]
  \[
  f_{\chi^{2}_{n}}(x) =
  \left\{
    \begin{array}{ll}
      \frac{e^{-\frac{x}{2}}x^{\frac{n}{2} - 1}}{2^{\frac{n}{2}}\Gamma\left(\frac{n}{2}\right)} & x > 0\\
      0 & x \le 0
    \end{array}
  \right.
  \]
  Deze verdeling bekomen we als we de afstand van een pijl tot het centrum van een schietroos bekijken in een $n$-dimensionale boogschietwedstrijd.
\end{de}

\subsubsection{Gammaverdeling}
\label{sec:gammaverdeling}

\begin{de}
  De \term{gammaverdeling} met parameters $\gamma, \beta \in \mathbb{R}^{+}$ is gedefinieerd door de dichtheidsfuncite $f_{\gamma,\beta}$:
  \[ f_{\gamma,\beta} = \frac{x^{\gamma-1}e^{-\frac{x}{\beta}}}{\beta^{\gamma}\Gamma(\gamma)}\mathbb{1}_{x > 0} \]
  \[ X \sim \Gamma_{\gamma,\beta} \]
\end{de}

\begin{opm}
  Deze verdeling wordt vaak gebruikt om wachttijden te modelleren.
\clarify{hoe dan?}
\end{opm}

\subsubsection{$t_n$ verdeling}
\label{sec:t_n-verdeling}

\begin{de}
  De \term{Student verdeling} met $n$ vrijheidsgraden is gedefinieerd als de verdeling van $T$:
  \[ T = \frac{Z}{\sqrt{\nicefrac{X}{n}}} \]
\end{de}

\begin{st}
  De verdeling is symmetrisch met dichtheid $f_{t_{n}}(t)$
  \[ f_{t_{n}}(t) = \frac{K_{n}}{\left(1+\frac{t^{2}}{n}\right)^{\frac{n+1}{2}}} \]
  Waarbij:
  \[ K_{n} = \frac{\Gamma\left(\frac{n+1}{2}\right)}{\sqrt{n\pi}\Gamma\left(\frac{n}{2}\right)} \]
\zb
\end{st}

\subsubsection{Cauchy verdeling}
\label{sec:cauchy-verdeling}

\begin{de}
  De \term{Cauchy verdeling} is de $t_{1}$ verdeling.
\end{de}

\begin{ei}
  Zij $X$ en $Y$ twee standaard normaal verdeelde stochastische variabelen, dan is $\nicefrac{X}{Y}$ Cauchy verdeeld.

  \begin{proof}
    \[ \nicefrac{X}{Y} \sim \nicefrac{X}{\sqrt{Y^{2}}} \sim t_{1} \]
  \end{proof}
\end{ei}

\begin{st}
  De dichtheidsfunctie van een Cauchyverdeling:
  \[ f(x) = \frac{1}{\pi}\frac{1}{1+x^{2}} \]
\end{st}

\begin{st}
  De verdelingsfunctie van en Cauchyverdeling:
  \[ \frac{1}{\pi}\arctan(x) + \frac{1}{2} \]
\end{st}

\subsubsection{$F$-verdeling}
\label{sec:f-verdeling}

\begin{de}
  Zij $ W\sim \chi_{m}^{2}$ en $V \sim \chi_{n}^{2}$ onafhankelijk, dan noemt men de verdeling van $X$ de \term{$F$-verdeling} van $m$ vrijheidsgraden in de teller en $n$ vrijheidsgraden in de noemer.
  \[ X = \frac{\nicefrac{W}{m}}{\nicefrac{V}{n}} \]
  \[ X \sim F_{m,n} \]
\end{de}

\begin{st}
  De dichtheidsfunctie van de $F_{m,n}$-verdeling is $f_{m,n}$
  \[
  f_{m,n} = 
  \left\{
    \begin{array}{ll}
      K_{m,n}\frac{x^{\frac{m}{2}-1}}{(n+mx)^{\frac{n+m}{2}}} & x > 0\\
      0 & x \le 0
    \end{array}
  \right.
  \]
  \[
  K_{m,n} = \frac{m^{\frac{m}{2}}n^{\frac{n}{2}}\Gamma\left(\frac{m+n}{2}\right)}{\Gamma\left(\frac{m}{2}\right)\Gamma\left(\frac{n}{2}\right)}
  \]
  \zb
\end{st}

\subsubsection{Lognormale verdeling}
\label{sec:lognormale-verdeling}

\begin{de}
  Een stochastische variabele is \term{lognormaal verdeeld} met parameters $\mu \in\mathbb{R}$ en $\sigma \in \mathbb{R}^{+}$ als het volgende geldt:
  \[ ln(Y) = N(\mu,\sigma^{2})\]
  't Is te zeggen, Als $X$ normaal verdeeld is, noemt men $e^{X}$ lognormaal verdeeld.
\end{de}

\section{Transformatie van stochastische veranderlijken}
\label{sec:transf-van-stoch}

\subsection{Monotone transformatie}
\label{sec:monot-transf}

\begin{st}
  Zij $X$ een continue stochastische veranderlijke met dichtheidsdfunctie $f_{X}$ zodat $f_{X}(x)=0$ voor $x\not\in S$. Zijn $h: \mathbb{R} \rightarrow \mathbb{R}$ een functie zodat $U=h(X)$ opnieuw een stochastische veranderlijke is. Zij bovendien $h$ een differentieerbare, strikt stijgende of strikt dalende functie op $S$, dan is de dichtheidsfunctie van $U$ $f_{U}$:
  \[
  f_{U}(u) = 
  \left\{
    \begin{array}{ll}
      f_{X}(h^{-1}(u)) \left| \frac{dh^{-1}(u)}{du} \right|& u\in h(S)\\
      0 & u \not \in h(S)
    \end{array}
  \right.
  \]
\extra{bewijs}
\end{st}

\subsection{Integraaltransformatie}
\label{sec:integr}

\begin{ei}
  Zij $X$ een continue stochastische veranderlijke met verdelingsfunctie $F_{X}$ strikt stijgend op $F_{X}^{-1}(]0,1[)$ en beschouw de transformatie $U=F_{X}(X)$, dan geldt $U \sim \mathcal{U}[0,1]$.
\extra{bewijs}
\end{ei}

\subsection{Genereren van stochastische veranderlijken}
\label{sec:gener-van-stoch}

\begin{ei}
  Zij $U \sim \mathcal{U}[0,1]$ en beschouw de transformatie $X = F^{-1}(U)$, dan is $F$ de verdelingsfunctie van $X$.
\extra{bewijs}
\end{ei}


\end{document}

%%% Local Variables:
%%% mode: latex
%%% TeX-master: t
%%% End:
