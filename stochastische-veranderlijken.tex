\documentclass[main.tex]{subfiles}
\begin{document}


\chapter{Stochastische Veranderlijken}
\label{cha:stoch-verand}

\section{Inleiding}
\label{sec:inleiding}

\begin{de}
  $\mathcal{B}(\mathbb{R})$ is de $\sigma$-algebra voortgebracht door $\mathcal{C} = \{ ]-\infty,a]\mid -\infty < a < +\infty \}$.
\end{de}

\begin{de}
  Zij $\Omega, \mathcal{A}, P$ een kansruimte.
  Een re\"ee functie $X$ als volgt noemen we een \term{stochastische veranderlijke}, een \term{stochastische variabele}, een \term{stochastiek}, een \term{toevalsvariabele} of een term{random variabele}.
  \[ X:\ \Omega \rightarrow \mathbb{R} \]
  \[ \forall B\in \mathcal{B}(\mathbb{R}):\ X^{-1}(B) = \{\omega \mid X(\omega) \in B \} \]
  We noemen $X$ dan een $\mathcal{A}$-\term{meetbare afbeelding}.
\end{de}

\begin{de}
  Een $\mathcal{B}(\mathbb{R})$-meetbare afbeelding in de meetbare ruimte $\mathbb{R},\mathcal{B}(\mathbb{R})$ noemen we \term{Borel-meetbaar}.
\end{de}

\begin{st}
  Een afbeelding $X: \mathbb{R} \rightarrow \mathbb{R}$ is een stochastische variabele in de meetbare ruimte $\mathbb{R},\mathcal{B}(\mathbb{R})$ als en slechts als het volgende geldt:
  \[ \forall a \in \mathbb{R}:\ X^{-1}(]-\infty,a]) = \{ \omega\mid X(\omega) \le a\} \in \mathcal{A} \]
  \zb
\end{st}

\begin{st}
  Een Borel-meetbare afbeelding introduceert een \term{kansmaat} $P_{X}(B)$ op $\mathcal{B}(\mathbb{R})$.
  \[ P_{X}(B) = P(X \in B) = P(X^{-1}(B)) \]
  \[ P_{X}(B) = P(\{\omega \in \Omega \mid X(\omega) \in B \}) \]
\extra{bewijs}
\end{st}

\begin{de}
  Zij $\Omega,\mathcal{A},P)$ een kansruimte en $X: \Omega \rightarrow \mathbb{R}$ een stochastische veranderlijke, dan definieren we de overeenkomstige \term{verdelingsfunctie} $F_{X}$ als volgt:
  \[ F: \mathbb{R} \rightarrow \mathbb{R}:\ F_{X}(a) = P_{X}(]-\infty,a]) = P(\{\omega\mid X(\omega) \le a\}) = P(X \le a) \]
\end{de}

\begin{st}
  Zij $X$ een stochastische verzameling in de kansruimte $\Omega,\mathcal{A},P$, dan is de afbeelding $F_{X}$ een verdelingsfunctie als de volgende beweringen gelden.
  \begin{enumerate}
  \item $F_{X}$ is monotoon stijgend.
    \[ \forall a,b\in \mathbb{R}:\ a\le b \Rightarrow F_{X}(a) \le F_{X}(b) \]
  \item $\lim_{a\rightarrow +\infty}F_{X}(a) = 1$ en $\lim_{a \rightarrow -\infty}F_{X}(a) = 0$
  \item $F_{X}$ is rechts continu.
    \[ \forall a\in\mathbb{R}:\ \lim_{h \overset{>}{\rightarrow} 0}F_{X}(a+h) = F_{X}(a) \]
  \end{enumerate}
\zb
\end{st}

\begin{st}
  Zij $F_{X}$ een verdelingsfunctie voor een kansruimte $\Omega,\mathcal{A},P)$.
  \[ P(a < X \le b) = F_{X}(b) - F_{X}(a) \]
\extra{bewijs}
\end{st}


\begin{st}
  Zij $F_{X}$ een verdelingsfunctie voor een kansruimte $\Omega,\mathcal{A},P)$.
  \[ P(X > a) = 1-F_{X}(a) \]
\extra{bewijs}
\end{st}

\begin{de}
  De \term{kwantielfunctie} $Q_{X}$ voor een kansruimte $\Omega,\mathcal{A},P)$ is de inverse functie van de verdelingsfunctie $F_{X}$.
  De waarde $Q_{X}(p)$ is de kleinste waarde $a$ waarvoor $F_{X}(a) \ge p$ geldt.
\end{de}

\begin{de}
  Het $25\%$, $50\%$ en $75\%$ kwantiel worden respectievelijk het eerste, tweede en derde \term{kwartiel} genoemd.
\end{de}

\begin{de}
  Het tweede kwartiel wordt ook wel de \term{mediaan} genoemd.
\end{de}

\section{Types stochastische veranderlijken}
\label{sec:types-stoch-verand}




\end{document}

%%% Local Variables:
%%% mode: latex
%%% TeX-master: t
%%% End:
