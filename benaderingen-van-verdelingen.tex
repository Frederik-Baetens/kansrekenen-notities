\documentclass[main.tex]{subfiles}
\begin{document}

\chapter{Benaderingen van verdelingen}
\label{cha:benad-van-verd}

\section{Limietstellingen}
\label{sec:limietstellingen}

\begin{st}
  De \term{centrale limietsstelling} (\term{CLT})\\
  Zij $X_{i}$ $n$ identiek verdeelde stochastische veranderlijken en $S = \sum_{i}X_{1}$.
  \[ \frac{S-n\mu}{\sqrt{n}\sigma} \rightarrow Z \]
  \[ \forall x \in \mathbb{R}:\ \lim_{n \rightarrow \infty} P\left(\frac{S - n\mu}{\sqrt{n}\sigma} \le x\right) = \Phi(x) \]
  \zb
\end{st}

\begin{st}
  De \term{stelling vna De Moivre-Laplace}\\
  Zij $X_{i}$ $n$ onafhankelijke stochastische veranderlijken die allemaal Bernoulli verdeeld zijn met parameter $p\in ]0,1[$ en $S = \sum_{i}X_{i}$.
  \[ \frac{S-n\mu}{\sqrt{n}\sigma} \rightarrow Z \]
  \TODO{bewijs p 94}
\end{st}

\begin{st}
  De \term{limietstelling van Poisson}\\
  Zij $X_{i}$ $n$ binomiaal verdeelde stochastische veranderlijken.
  Als $\lim_{n \rightarrow +\infty, p_{i} \rightarrow 0}np_{i} = \alpha$ geldt, dan ook:
  \[
  X_{i} \rightarrow X \text{ met } X \sim \mathcal{P}(\alpha)
  \]
\end{st}


\section{Practische benaderingen}
\label{sec:pract-benad}



\end{document}

%%% Local Variables:
%%% mode: latex
%%% TeX-master: t
%%% End:
