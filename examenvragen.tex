\documentclass[main.tex]{subfiles}
\begin{document}


\chapter{Examenvragen}
\label{cha:examenvragen}

\begin{examenvraag}{Tussentijdse Toets 2014}
  \begin{ex-vraag}
    Een stochastisch koppel $(X,Y)$ heeft een bivariate normale verdeling als de gezamelijke verdelingsfunctie van $X$ en $Y$ als volgt gegeven wordt:
    \[
    f_{X,Y}(x,y)
    =
    \frac
    {
      e^{-\frac{1}{2}\frac{1}{1-\rho^{2}}
        \left(
          \left(\frac{x-\mu_{x}}{\sigma_{x}}\right)^{2}
          - 2\rho\left(\frac{x-\mu_{x}}{\sigma_{x}}\right)^{2}\left(\frac{y-\mu_{y}}{\sigma_{y}}\right)^{2}
          + \left(\frac{y-\mu_{y}}{\sigma_{y}}\right)^{2}
        \right)
      }
    }
    {
      2\pi\sigma_{x}\sigma_{y}\sqrt{1-\rho^{2}}
    }
    \]
    Bereken de marginale dichtheidsfunctie van $Y$.
  \end{ex-vraag}
  \begin{ex-antwoord}
    \kennen{De formule voor een marginale dichtheidsfunctie gaat als volgt:}
    \[ f_{y}(y) = \int_{-\infty}^{+\infty}f_{X,Y}(x,y)\ dx \]
    We voeren eerst een aantal constanten in, om de leesbaarheid te vergroten:
    \[ 
    C = \dfrac{1}{2\pi\sigma_{x}\sigma_{y}\sqrt{1-\rho^{2}}} \quad A = \frac{1}{2}\frac{1}{1-\rho^{2}} \quad u = \frac{x-\mu_{x}}{\sigma_{x}} \quad v = \frac{y-\mu_{y}}{\sigma_{y}}
    \]
    \[ f_{X,Y}(x,y) = C e^{A \cdot (u^{2}-2\rho u v + v^{2})} \]
    \kunnen{We kunnen de formule nu uitrekenen}
    \[ \int_{-\infty}^{+\infty}f_{X,Y}(x,y)\ dx\ dy = \int_{-\infty}^{+\infty} C e^{-A \cdot (u^{2}-2\rho u v + v^{2})}\ dx = C\int_{-\infty}^{+\infty}e^{-A \cdot (u^{2}-2\rho u v + v^{2})}\ dx \]
    We voeren een nieuwe integratieveranderlijke in:
    \[ 
    \begin{array}{rl}
      u &= \frac{x-\mu_{x}}{\sigma_{x}}\\
      du &= \frac{1}{\sigma_{x}}dx\\
    \end{array}
    \]
    \[
    = C\sigma_{x}\int_{-\infty}^{+\infty}e^{-A \cdot (u^{2}-2\rho u v + v^{2})}\ du
    = C\sigma_{x}\sigma_{X}\int_{-\infty}^{+\infty}e^{-Av^{2}}e^{-A \cdot (u^{2}-2\rho u v)}\ du
    = C\sigma_{x}e^{-Av^{2}}\int_{-\infty}^{+\infty}e^{-Au^{2}+2A\rho u v}\ du
    \]
    We forceren hier een bijzonder product door $+ A\rho^{2}v^{2} - A\rho^{2}v^{2}$ toe te voegen in de exponent.
    \[
    = C\sigma_{x}e^{-Av^{2}}\int_{-\infty}^{+\infty}e^{-Au^{2}+2A\rho u v + A\rho^{2}v^{2} - A\rho^{2}v^{2}}\ du 
    = C\sigma_{x}e^{-Av^{2}-A\rho^{2}v^{2}}\int_{-\infty}^{+\infty}e^{-Au^{2}+2A\rho u v + A\rho^{2}v^{2}}\ du 
    \]
    \[
    = C\sigma_{x}e^{-A(1-\rho^{2})v^{2}}\int_{-\infty}^{+\infty}e^{-A(u-\rho v)^{2}}\ du 
    \]
    We voeren nogmaals een nieuwe integratieveranderlijke in:
    \[ 
    \begin{array}{rl}
      z &= \sqrt{2A}(u-\rho v) \\
      dz &= \sqrt{2A} du\\
    \end{array}
    \]
    $z^{2}$ is dan $2A(u-\rho v)^{2}$.
    \[
    = C\sigma_{x}e^{-A(1-\rho^{2})v^{2}}\frac{1}{\sqrt{A}}\int_{-\infty}^{+\infty}e^{-\frac{1}{2}z^{2}}\ dz
    \]
    Nu zou je de formule binnen de integraal moeten herkennen.
    \formularium{Dichtheidsfunctie van een algemene Normaalverdeling}{ $\frac{e^{-\frac{(x-\mu)^{2}}{2\sigma^{2}}}}{\sigma\sqrt{2}\pi}$}
    De formule voor de standaard normaalverdeling ziet er als volgt uit:
    \[ \phi(x) = \frac{e^{-\frac{1}{2}x^{2}}}{\sqrt{2\pi}} \]
    Omdat dit een dichtheid is, moet de integraal tot $1$ evalueren:
    \[ 1 = \int_{-\infty}^{+\infty}\frac{e^{-\frac{1}{2}x^{2}}}{\sqrt{2\pi}} \]
    \[ \sqrt{2\pi} = \int_{-\infty}^{+\infty}e^{-\frac{1}{2}x^{2}} \]
    De integraal die nog overblijft kunnen we dus vervangen door $\sqrt{2\pi}$.
    \[
    = C\sigma_{x}e^{-A(1-\rho^{2})v^{2}}\frac{\sqrt{2\pi}}{\sqrt{2A}}
    \]
    We vervangen nu de leesbaarheidsconstanten terug naar de originele waarden en zien dat we al veel kunnen vereenvoudigen.
    \[
    = \dfrac{1}{2\pi\sigma_{x}\sigma_{y}\sqrt{1-\rho^{2}}}\sigma_{x}e^{-\left( \frac{1}{2}\frac{1}{1-\rho^{2}}\right)(1-\rho^{2})\left(\frac{y-\mu_{y}}{\sigma_{y}}\right)^{2}}\frac{\sqrt{2\pi}}{\sqrt{2\left( \frac{1}{2}\frac{1}{1-\rho^{2}}\right)}}
    \]
    \[
    = \dfrac{1}{\sqrt{2\pi}\sigma_{y}}e^{- \frac{1}{2}\left(\frac{y-\mu_{y}}{\sigma_{y}}\right)^{2}}
    \]
    We vinden dat $f_{y}(y)$ de dichtheidsfunctie is van een normaal verdeelde stochastische variabele, zoals verwacht.
  \end{ex-antwoord}
\end{examenvraag}

\begin{examenvraag}{Tussentijdse Toets 2014}
  \begin{ex-vraag}
  Zij $\Omega$ een niet-aftelbare verzameling en zij $\mathcal{A}$ de volgende verzameling:
  \[ \mathcal{A} = \{ A \in \Omega \mid A \text{ is aftelbaar} \vee A^{C} \text{ is aftelbaar } \} \]
  Bewijs dat $\mathcal{A}$ een $\sigma$-algebra is.
  \end{ex-vraag}
  \begin{ex-antwoord}
    \begin{proof}
      \kennen{We gaan eenvoudigweg alle definierende eigenschappen van een $\sigma$-algebra af:}
      \begin{itemize}
      \item $\Omega \in \mathcal{A}$\\
        $\Omega^{C}$ is leeg en dus aftelbaar, dus $\Omega$ zit in $\mathcal{A}$.
      \item $\forall A \in \mathcal{A}: A^{C}\in \mathcal{A}$\\
        Kies een willekeurige $A \in \mathcal{A}$. 
        We onderscheiden twee gevallen.
        \begin{itemize}
        \item $A$ is aftelbaar: $A^{C}$ zit dan in $\mathcal{A}$ want $A^{C^{C}} = A$ is aftelbaar.
        \item $A^{C}$ is aftelbaar: $A$ zit dan in $\mathcal{A}$ want $A^{C}$ is aftelbaar.
        \end{itemize}
      \item $\forall (A_{n})_{n}, A\in \mathcal{A}: \bigcup_{n} A_{n} \in \mathcal{A}$\\
        Fixeer een rij $(A_{n})_{n}$ in $\mathcal{A}$.
        We onderscheiden opnieuw twee gevallen.
        \begin{itemize}
        \item Alle $A_{n}$ zijn aftelbaar: $\bigcup_{n}A_{n}$ is dan aftelbaar en zit dus in $\mathcal{A}$.
        \item Er is minstens \'e\'en $A_{n}$ niet aftelbaar. Opdat die $A_{m}$ in $\mathcal{A}$ zou zitten moet $A_{m}^{C}$ aftelbaar zijn.
          $\left(bigcup_{n}A_{n}\right)^{C} = \bigcap_{n}A_{n}^{C}$ is nu aftelbaar omdat $A_{m}$ aftelbaar is.
          Bijgevolg zit $\bigcup_{n}A_{n}$ ook in $\mathcal{A}$.
        \end{itemize}
      \end{itemize}
    \end{proof}
  \end{ex-antwoord}
\end{examenvraag}

\begin{examenvraag}{Tussentijdse Toets 2014}
  \begin{ex-vraag}
    Yumm's is een restaurant dat ook aan huis levert.
    Op hun website kan je terugvinden dat de gemiddelde wachttijd (tussen het moment waarop je bestelt en het moment waarop het eten geleverd wordt) $60$ minuten bedraagt, met als standaarddeviatie $20$ minuten.
    Ook restaurant Tasty levert aan huis, gemiddeld gezien met een wachttijd van $75$ minuten met een standaarddeviatie van $10$ minuten.
    Wanneer je eten bestelt, doe je dat $6$ van de $10$ keer bij Yumm's, en $4$ van de $10$ keer bij Tasty.
    Indien je veronderstelt dat beide wachttijden normaal verdeeld zijn, geef dan antwoord op de volgende vragen:
    \begin{itemize}
    \item Indien je een bestelling geplaatst hebt om $19$ uur, en je eten wordt geleverd tussen $20.15$ uur en $20.30$ uur, hoe groot is dan de kans dat je bij Yumm's besteld hebt?
    \item Indien je elke week bij Yumm's bestelt, hoe groot is dan de kans dat je eten gedurende \'e\'en jaar ($52$ weken) minstens $10$ keer tussen $20.15$ uur en $20.30$ uur geleverd werd?
    \end{itemize}
  \end{ex-vraag}
  \begin{ex-antwoord}
    Noem $X$ de wachttijd wanneer je eten bestelt bij Yumm's.
    We gaan ervan uit dat $X$ normaal verdeeld is.
    \[ X \sim N(\mu_{X}=60,\sigma_{X}=20) \]
    Noem $Y$ de wachttijd wanner je eten bestelt bij Tasty.
    We gaan ervan uit dat ook $Y$ normaal verdeeld is.
    \[ Y \sim N(\mu_{Y}=75,\sigma_{Y}=10) \]
    We gaan er ook van uit dat je enkel bij Yumm's of Tasty kan bestellen.
    \begin{itemize}
    \item 
      Merk op dat de wachttijd tussen $75$ en $90$ minuten ligt in dit geval.
      Noem $A$ de gebeurtenis waarin de wachttijd tussen $75$ en $90$ minuten ligt
      Noem $B$ de gebeurtenis waarin je bij Yumm's bestelt.
      We zoeken nu de volgende kans:
      \[ P(B|A) \]
      \kennen{We gebruiken de stelling van Bayes}:
      \[ 
      P(B|A) = \frac{P(A|B)P(B)}{P(A|B)P(B) + P(A|B^{C})P(B^{C})}
      \]
      Hierin kennen we $P(B)$, $P(B^{C})$ en kunnen we $P(A|B)$ en $P(A|B^{C})$ berekenen.
      \begin{itemize}
      \item $P(A|B)$:\\
        $P(A|B)$ is de kans op de gegeven wachttijd, gegeven dat we bij Yumm's bestellen.
        \[ P(A|B) = P(75 \le X \le 90) \]
        Omdat $X$ normaal verdeeld is, kunnen we deze kans berekenen met behulp van onze tabellen.
        \[ P(75 \le X \le 90) = P(\frac{75-60}{20} \le Z \le \frac{90-60}{20}) = P(Z \le 1.5) - P(Z \le 0.75)\]
        \tabel{Standaard normale verdeling}{$P(Z \le 0.75)=0.733$ en $P(Z \le 1.5) = 0.933$}
        \[ = 0.933 - 0.773 = \frac{4}{25} = 0.16 \]
      \item $P(A|B^{C})$:\\
        $P(A|B^{C})$ is de kans op de gegeven wachttijd, gegeven dat we bij Tasty bestellen.
        \[ P(A|B^{C} = P(75 \le Y \le 90) \]
        Omdat $Y$ normaal verdeeld is, kunnen we deze kans berekenen met behulp van onze tabellen.
        \[ P(75 \le Y \le 90) = P(\frac{75-75}{10} \le Z \le \frac{90-75}{10}) = P(Z \le 1.5) - P(Z \le 0) \]
        \tabel{Standaard normale verdeling}{$P(Z \le 1.5) = 0.773$ en $P(Z \le 0) = 0.5$}
        \[ = 0.933 - 0.5 = \frac{433}{1000} = 0.433 \]
      \end{itemize}
      We kunnen nu de gevraagde kans uitrekenen:
      \[ \frac{P(A|B)P(B)}{P(A|B)P(B) + P(A|B^{C})P(B^{C})} = \frac{\frac{4}{25}\frac{6}{10}}{\frac{4}{25}\frac{6}{10} + \frac{433}{1000}\frac{4}{10}} = \frac{240}{673} = 0.3566 \]

    \item 
      We gaan ervan uit dat je telkens om $19$ uur bestelt.
      Noem $U$ het aantal keer dat het eten tussen $20.15$ uur en $20.30$ uur toekomt per jaar.
      De kans dat het eten \'e\'en keer tussen $20.15$ uur en $20.30$ uur toekomt is $0.16$ (zie eerder).
      \kennen{$U$ is dan binomiaal verdeeld met $p=0.16$}.
      \[ U \sim B(n=52, p=0.16) \]
      We zoeken nu de volgende kans:
      \[ P(U > 9) \]
      We kunnen stukje bij beetje uitrekenen door $42$ termen uit te schrijven en uit te rekenen, of we kunnen de centrale limietstelling gebruiken:
      \[ U = 52 X \sim N(52\mu_{X},52\sigma_{X}^{2}) \]
      We hebben hiervoor $\mu_{X}$ en $\sigma_{X}^{2}$ nodig.
      \kennen{Aangezien $X$ twee mogelijke uitkomsten heeft, in de zin dat het eten ofwel binnen de grenzen, ofwel buiten de grenzen aankomt, is $X$ Bernoulli-verdeeld met $p=0.16$.}
      \formularium{Bernoulli verdeling}{$\mu_{X}=p$ en $\sigma_{X}=p(1-p)$}
      \[ U = 52 X \sim N(\mu_{U}= 52\cdot 0.16,\ \sigma_{U} = 52\cdot 0.16 \cdot 0.84) = N(\mu_{U}=8.32,\sigma_{U}=6.99) \]
      We kunnen de gezochte kans nu berekenen met de tabellen.
      \[ P(U \ge 10) = 1 - P(U \le 10) = 1 - P\left(Z \le \frac{10-0.5-8.32}{\sqrt{6.99}} \right) = 1 - P(Z \le 0.45) \]
      \tabel{Standaard normale verdeling}{$P(Z \le 0.674) = 0.326$}
      \[ = 1-0.674 = 0.326 \]
    \end{itemize}
  \end{ex-antwoord}
\end{examenvraag}


\begin{examenvraag}{Tussentijdse Toets 2014}
  \begin{ex-vraag}
    Beschouw een toevalsvariabele $X$ met de volgende cumulatieve verdelingsfunctie.
    \[ F_{X}(x) = 1 - e^{-\left(\frac{x}{\alpha}\right)^{\beta}} \quad x \ge 0 \]
    Met parameters $\alpha$ en $\beta$.
    Deze verdelingsfunctie beschrijft de Weibull verdeling.
    \begin{itemize}
    \item Bepaal de dichtheidsfunctie van $X$.
    \item Stel $Y = \left(\frac{X}{\alpha}\right)^{\beta}$, wat kan je vertellen over de verdeling van $Y$?
    \item Hoe kunnen we stochastische veranderlijken komende van de Weibull verdeling genereren, vertrekkende van $U \sim U\interval{0}{1}$? 
    \end{itemize}
  \end{ex-vraag}
  \begin{ex-antwoord}
    \begin{itemize}
    \item
      \kennen{De dichtheidsfunctie van $X$ is de afgeleide van de cumulatieve verdelingsfunctie.}
      \[ f_{X} = \frac{dF_{X}(x)}{dx} = e^{-\left(\frac{x}{\alpha}\right)^{\beta}}\frac{\beta}{\alpha^{\beta}}x^{\beta-1} \]
      \needed
    \item 
      Noem $g$ de functie die $X$ op $Y$ afbeeldt:
      \[ g:\ \mathbb{R}^{+} \rightarrow \mathbb{R}^{+}:\ x \mapsto y = g(x) = \left(\frac{x}{\alpha}\right)^{\beta} \]
      De inverse functie noemen we $h$:
      \[ h = g^{-1}: y \mapsto x = h(y) = \alpha\sqrt[\beta]{y} \]
      \kennen{We kunnen nu $f_{Y}$ berekenen vanuit $f_{X}$:}
      \[ f_{Y}(y) = f_{X}(h(y))|h'(y)| = e^{-\left(\frac{\alpha\sqrt[\beta]{y}}{\alpha}\right)^{\beta}}\frac{\beta}{\alpha^{\beta}}\left(\alpha\sqrt[\beta]{y}\right)^{\beta-1} \frac{\alpha}{\beta}y^{\frac{1-\beta}{\beta}} = e^{-y} \]
      \formularium{$\mathcal{E}$}{$f(x)$ voor $\mathcal{E}(1)$ is $e^{-x}$ voor $x>0$.}
      Nu blijkt het volgende:
      \[ X \sim \mathcal{E}(1) \]
      \clarify{meer uitleg, waarom deze werkwijze?}
    \item Gegeven $U \sim U\interval{0}{1}$:
      \[
      \begin{array}{rl}
        F_{X} &= 1 - e^{-\left(\frac{x}{\alpha}\right)^{\beta}}\\
        z &= 1 - e^{-\left(\frac{x}{\alpha}\right)^{\beta}}\\
        x &= a\sqrt[\beta]{-\ln(1-z)}\\
        F_{X}^{-1} &= a\sqrt[\beta]{-\ln(1-z)}\\
      \end{array}
      \]
      \clarify{meer uitleg, waarom deze werkwijze?}
    \end{itemize}
  \end{ex-antwoord}
\end{examenvraag}

\begin{oef}{Tussentijdse Toets 2014}
  \begin{ex-vraag}
    Zij $X_{1}$ een exponentieel-verdeelde stochastische variabele met dichtheid $f_{X_{1}}$: 
    \[ f_{X_{1}}(x) = \frac{1}{6}e^{-\frac{x}{6}} \quad x \ge 0 \]
    Zij $X_{2}$ een een $\chi^{2}$ verdeelde stochastische variabele met dichtheid $f_{X_{2}}$: 
    \[ f_{X_{2}}(x) = \frac{e^{-\frac{x}{2}}x^{2-1}}{2^{2}\Gamma(2)} \]
    Wat is de verdeling van $X_{1} + 3X_{2}$ als $X_{1}$ en $X_{2}$ onafhankelijk zijn?
  \end{ex-vraag}
  \begin{ex-antwoord}
    We bekijken de momentgenererende functies van $X_{1}$ en $X_{2}$.
    Omdat $X_{1}$ en $X_{2}$ onafhankelijk zijn, is de momentgenerenede functie van $X_{1} + 3X_{2}$ eenvoudig te berekenen:
    \[ M_{X_{1}+3X_{2}}(t) = M_{X_{1}}(t)M_{3X_{2}}(t) = M_{X_{1}}(t)M_{X_{2}}(3t) \]
    \formularium{Exponentiele verdeling}{$M_{X_{1}}(t) = \frac{1}{\sqrt{1-6t}}$ voor $t < \alpha$ ($\alpha = \nicefrac{1}{6}$)}
    \formularium{$\chi^{2}$-verdeling}{$M_{X_{2}}(t) = \frac{1}{1-2t} $ ($n = 4$)}
    We vinden de momentgenererende functie:
    \[ M_{X_{1}+3X_{2}}(t) = \frac{1}{\sqrt[3]{1-6t}} \]
    Kijken we nu terug naar het formularium, dan vinden we de verdeling van $X_{1} + 3X_{2}$.
    \formularium{$\Gamma$-verdeling}{$M_{\Gamma_{3,6}}= \frac{1}{\sqrt[3]{1-6t}}$}
    \[ X_{1} + 3X_{2} \sim \Gamma(3,6) \]
  \end{ex-antwoord}
\end{oef}

\end{document}

%%% Local Variables:
%%% mode: latex
%%% TeX-master: t
%%% End:
